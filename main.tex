\documentclass[acmsmall]{acmart}
\usepackage{cleveref,xspace}
\usepackage{mathpartir,xcolor}
\newcommand{\sLet}{\text{\textsf{Let}}\xspace}
\newcommand{\sLam}{\text{\textsf{Lam}}\xspace}
\newcommand{\sApp}{\text{\textsf{App}}\xspace}
\newcommand{\sProd}{\text{\textsf{Prod}}\xspace}
\newcommand{\sZro}{\text{\textsf{Zro}}\xspace}
\newcommand{\sFst}{\text{\textsf{Fst}}\xspace}
\newcommand{\sLeft}{\text{\textsf{Left}}\xspace}
\newcommand{\sRight}{\text{\textsf{Right}}\xspace}
\newcommand{\sCase}{\text{\textsf{Case}}\xspace}
\newcommand{\sIn}{\text{\textsf{in}}\xspace}
\newcommand{\sOf}{\text{\textsf{of}}\xspace}

\newcommand{\Env}{\ensuremath{\mathit{Env}}\xspace}
\newcommand{\KCell}{\text{\textnormal{KCell}}\xspace}
\newcommand{\VCell}{\text{\textnormal{VCell}}\xspace}
\newcommand{\KLet}{\text{\textsf{KLet}}\xspace}
\newcommand{\KLookup}{\text{\textsf{KLookup}}\xspace}
\newcommand{\Done}{\text{\textsf{Done}}\xspace}
\newcommand{\KApp}{\text{\textsf{KApp}}\xspace}
\newcommand{\KProd}{\text{\textsf{KProd}}\xspace}
\newcommand{\KZro}{\text{\textsf{KZro}}\xspace}
\newcommand{\KFst}{\text{\textsf{KFst}}\xspace}
\newcommand{\KLeft}{\text{\textsf{KLeft}}\xspace}
\newcommand{\KRight}{\text{\textsf{KRight}}\xspace}
\newcommand{\KCase}{\text{\textsf{KCase}}\xspace}
\newcommand{\Clos}{\text{\textsf{Clos}}\xspace}
\newcommand{\VProd}{\text{\textsf{VProd}}\xspace}
\newcommand{\VLeft}{\text{\textsf{VLeft}}\xspace}
\newcommand{\VRight}{\text{\textsf{VRight}}\xspace}
\newcommand{\Eval}{\text{\textsf{Eval}}\xspace}
\newcommand{\Apply}{\text{\textsf{Apply}}\xspace}

\newcommand{\RKApply}{\text{\textsf{RKApply}}\xspace}
\newcommand{\RHApply}{\text{\textsf{RHApply}}\xspace}
\newcommand{\RHCase}{\text{\textsf{RHCase}}\xspace}
\newcommand{\RHZro}{\text{\textsf{RHZro}}\xspace}
\newcommand{\RHFst}{\text{\textsf{RHFst}}\xspace}
\newcommand{\RHApp}{\text{\textsf{RHApp}}\xspace}
\newcommand{\RApply}{\text{\textsf{RApply}}\xspace}

\newcommand{\Just}{\textrm{Just}}
\newcommand{\Nothing}{\textrm{Nothing}}

\newcommand{\Alloc}{\textrm{Alloc}}
\newcommand{\Lookup}{\textrm{Lookup}}

\newcommand{\Insert}{\text{\textsf{Insert}}\xspace}

\newcommand{\NoReplay}{\text{\textsf{NoReplay}}\xspace}
\newcommand{\Replaying}{\text{\textsf{Replaying}}\xspace}

\setcopyright{acmlicensed}
\copyrightyear{2018}
\acmYear{2018}
\acmDOI{XXXXXXX.XXXXXXX}


\acmJournal{JACM}
\acmVolume{37}
\acmNumber{4}
\acmArticle{111}
\acmMonth{8}

\newcommand\todo[1]{\textcolor{red}{#1}}
\newcommand\pavel{\todo{pavel look here} }

\begin{document}
\title{Uncomputation}

\author{Marisa Kirisame}
\email{marisa@cs.utah.edu}
\orcid{1234-5678-9012}
\authornotemark[1]
\affiliation{%
	\institution{University of Utah}
	\streetaddress{P.O. Box 1212}
	\city{Salt Lake City}
	\state{Utah}
	\country{USA}
	\postcode{43017-6221}
}

\author{Heng Zhong}
\email{hzhong21@m.fudan.edu.cn}
\orcid{???}
\affiliation{%
	\institution{Fudan University}
	\streetaddress{220 Handan Road}
	\city{Yangpu District}
	\state{Shanghai}
	\country{P. R. China}
	\postcode{200437}
}

\author{Tiezhi Wang}
\email{2152591@tongji.edu.cn}
\orcid{???}
\affiliation{%
	\institution{Tongji University}
	\streetaddress{4800 Caoan Road}
	\city{Jiading District}
	\state{Shanghai}
	\country{P. R. China}
	\postcode{201804}
}

\author{Sihao Chen}
\email{21201112@hdu.edu.cn}
\orcid{???}
\affiliation{%
	\institution{Hangzhou Dianzi University}
	\streetaddress{No. 2 Street, Qiantang District}
	\city{Hangzhou}
	\state{Zhejiang}
	\country{P. R. China}
	\postcode{310018}
}

\author{Pavel Panchekha}
\email{}
\orcid{1234-5678-9012}
\authornotemark[1]
\affiliation{%
	\institution{University of Utah}
	\streetaddress{P.O. Box 1212}
	\city{Salt Lake City}
	\state{Utah}
	\country{USA}
	\postcode{43017-6221}
}

\renewcommand{\shortauthors}{Kirisame et al.}

\begin{abstract}
	Program execution need memory. Program may run out of memory for multiple reasons: big dataset, exploding intermediate state, the machine have less memory than others, etc. When this happens, the program either get killed, or the operating system swaps, significantly degrading the performance.
	We propose a technique, uncomputation, that allow the program to continue running gracefully even after breaching the memory limit, without significant performance degradation.
	Uncomputation work by turning computed values back into thunk, and upon re-requesting the thunk, computing and storing them back.
	A naive implementation of uncomputation will face multiple problems. Among them, the most crucial and the most challenging one is that of breadcrumb. After a value is uncomputed, it's memory can be released but some memory, breadcrumb, is needed, so we can recompute the value back.
	Ironically, in a applicative language, due to boxing all values are small. This mean uncomputation, implemented naively, will only consume more memory, defeating the purpose.
	We present a runtime system, implemented as a library, that is absolved of the above breadcrumb problem.
\end{abstract}

%%
%% Keywords. The author(s) should pick words that accurately describe
%% the work being presented. Separate the keywords with commas.
\keywords{Do, Not, Us, This, Code, Put, the, Correct, Terms, for,
	Your, Paper}

\maketitle


\section{Intro}
% \subsection{Motivation}
% Program execution consume both space and time. While there are tons of research focused on reducing the time spent running a program, memory usage reduction had been consistently underappreciated.

% Yet, saving memory is still an important topic worth studying:
% \begin{itemize}
% 	\item Multi-tenancy. A end-user laptop execute multiple programs concurrently. As an example, a typical programmer might open an IDE to edit and compile code, Zoom for remote meeting, and additionally a browser with dozens of tabs open to lookup information on the internet. Among them, the worst is the web browser, as each tab is it's own separate process, with a renderer, a rule engine, along side a JavaScript runtime. All software above consume a decent amount of ram, and contend between themselves for memory.
% 	\item Huge input. While a typical image might have a resolution of around 1000 x 1000, large images might reach a size 100 to 1000 time larger than that. Images editors and viewers typically thrash or oom upon processing images of such size. Similarly, text editors typically process text of < 1MB, and fail on logs or other large text file reaching GBs. Likewise IDEs will parse and analyze the source code of a project, then stored the analyzed results for auto completion. Upon working on huge project such IDEs will crash.
% 	\item Intermediate state. A modern computer can read and write memory at a speed of GB per seconds. Without memory reclamation technique such as garbage collection, memory will run out at a matter of seconds. However, some applications is essentially out of reach from garbage collection, and must keep most if not all intermediate states around. These applications include time travelling debugger, jupyter notebook, reverse mode automatic differentiation for deep learning, incremental computation/hash consing, and algorithm that use search, such as model checker and chess bot.
% 	\item Low memory limit. Wasm have a memory limit of 2GB, and embedded device or poor people may have device with lower ram. GPU and other accelerator use their own separate memory and usually is of smaller capacity then main memory.
% \end{itemize}

% A typical and generic solution to reduce memory consumption is by uncomputation. Uncomputation trade space for time, by tagging values with the metadata that created it(thunk), and when memory is low, deallocate values which might be used later on, recomputing them back with the thunk when the value is needed again.

% Another generic solution is swapping, where upon low memory, pages are swapped to disk and swapped back when needed again. While swapping had been implemented for basically all operating systems, it is an especially bad solution for functional and object-oriented languages, as those languages allocate and deallocate lots of small object, yet swapping work on the granularity of pages, unable to separate out the hot objects, that had been recently accessed from the code objects, that had not been accessed.

% Typically, uncomputation is implemented in an ad-hoc, case by case basis. When a software consume more memory than desired, it's programmer can (1) look at the program, (2) decide which part is using excessive memory, then (3) add custom data type to record the thunk, alongside the code to replace it (4) implement a cache eviction policy to decide what data to evict(uncompute). Another solution is to analyze the algorithm carefully, replacing the algorithm with a specialized version with recomputation in mind, see gradient checkpointing/iterative deepinging depth first search/island algorithm.

% Needless to say, this process is extremely cumbersome, taking precious developers time away from more critical task such as optimizing cpu usage or implementing new features, so a generic, automatic approach is needed to reduce memory consumption.

% \subsection{Failed Attempt}
% On first glance, uncomputation is easy, as it's cousin, lazy evaluation, is heavily studied. In lazy computation, object might be represented as thunks, which, when needed, is then computed, and the thunk is replaced by the value. A solution to uncomputation is to modify thunk, so that, upon computing, the old function is still kept. This allow the thunk to uncompute as if it had never been recomputed.

% \begin{mathpar}
% 	$Lazy Evaluation: type 'a lazy = MkLazy of ('a, unit -> 'a) either ref \\
% 	Uncomputation: type 'a uncompute = MkUncompute of 'a option ref * (unit -> 'a)$
% \end{mathpar}

% Just like lazy evaluation, we can then uniformally lift all values on top of our modified value type, inserting code that force weak head normal form upon data access, and have some cache policy to decide what to uncompute, which merely rewrite the reference back into the empty optional value.

% However, there are multiple problems with this approach:
% \begin{enumerate}
% 	\item size measurement. We want to gather statistics to decide what values to evict, and one of the most important statistics is the memory a object consume. However, the heap form a complex object graph, and fast cardinality estimation is highly complex.
% 	\item dopple ganger. The translation will lift a list into nested uncompute. Now suppose variable X hold a list, while variable Y hold a sublist of X, sharing the same representation\todo{bad wording dont know how to fix}. When X is uncomputed and recomputed, the corresponding Y part will be duplicated, so there will be two representation of Y in memory. In the extreme case, such duplication can force the program to take exponentially more memory. 
% 	\item bread crumb. A thunk capture other value as free variable, which contain a thunk part, capturing more value. This recursive process capture all value that the current value transitively compute on. Since we still need memory to store the thunk itself, and since a thunk is typically larger then an object, anything we gained on uncomputation, will be offset by the metadata.
% \end{enumerate}

% While the above three problems seems difficult, we can observe they are all but mere manifestation of recursiveness. In particular, size measurement and dopple ganger deal with recursive objects, and bread crumb deal with recursive thunk. This indicate that a solution that handle recursiveness well naturally solve the three problems at once.

% \subsection{A recursive Solution}
% To combat recursiveness we take heavy inspiration from Abstracting Abstract Machine (AAM). AAM lift abstract interpretation onto programming languages with arbitrary feature. The critical problem there is likewise recursiveness, as a naive lifting might allow the lattice infinite merge, causing the analysis to non-terminate. AAM propose to use an abstract machine that abstract over pointers, allocations and lookups to manage recursiveness.

% Likewise, the three recursiveness problem listed above can be handled by a similar manner. We started from a purely functional language, specifying it's semantic with the CEK machine. We can then abstract over pointers/allocations/lookups similarly.

% Most importantly, our key insight is that by a careful handling of our core data structure, the tock tree, which will be explained later, we can remove a value, alongside the metadata(thunk) on that value, yet still recompute it later. This solve the breadcrumb problem which render the naive apporach unsalvageable.

% Alongside our solution is proof that it is both correct and efficient. Our correctness proof state that uncomputing will not effect the final result(preservation), and will not infinite loop(progress). What get computed by the raw, uncompute-free semantic, no matter what we evicted. Our efficiency proof state that if our memory consumption is O(n), where n is the amount of live objects: objects that take O(1) time to force into weak head normal form. In particular, this imply asymptotically we use no space to store any evicted object, yet we somehow can access the metadata that recompute them!

% We had implemented our solution, alongside a eviction policy that is both fast to compute, and make better decision over classical eviction policy like LRU, random, and GDSF.

\subsection{Motivation}
Program execution consumes both space and time. While there is tons of research focused on reducing the time spent running a program, memory usage reduction has been consistently underappreciated.

Yet, saving memory is still an important topic worth studying:
\begin{itemize}
    \item Multi-tenancy. An end-user laptop executes multiple programs concurrently. As an example, a typical programmer might open an IDE to edit and compile code, Zoom for remote meetings, and additionally a browser with dozens of tabs open to look up information on the internet. Among them, the worst is the web browser, as each tab is its own separate process, with a renderer, a rule engine, and a JavaScript runtime. All software above consume a decent amount of RAM and contend between themselves for memory.
    \item Huge input. While a typical image might have a resolution of around 1000 x 1000, large images might reach a size 100 to 1000 times larger than that. Images editors and viewers typically thrash or oom upon processing images of such size. Similarly, text editors typically process text of < 1MB, and fail on logs or other large text files reaching GBs. Likewise, IDEs will parse and analyze the source code of a project and then store the analyzed results for auto-completion. Upon working on a huge project such IDEs will crash.
    \item Intermediate state. A modern computer can read and write memory at a speed of GB per second. Without memory reclamation techniques such as garbage collection, memory will run out in a matter of seconds. However, some applications are essentially out of reach from garbage collection and must keep most if not all intermediate states around. These applications include time traveling debugger, Jupyter Notebook, reverse mode automatic differentiation for deep learning, incremental computation/hash consing, and algorithms that use search, such as model checker and chess bot.
    \item Low memory limit. Wasm has a memory limit of 2GB, and embedded devices or poor people may have devices with lower RAM. GPU and other accelerator use their own separate memory and usually are of smaller capacity than main memory.
\end{itemize}

At best, programmers today approach memory savings
  in an ad-hoc, case-by-case way:
  they examine their program
  and rewrite it to use less memory,
  possibly by using caching, lazy computation, or novel algorithms.
However, these case-by-case solutions
  are not always available,
  since they depend on the details of the computation.
Moreover, even when available, 
  such solutions can require substantial programmer effort,
  including rewriting the program data flow
  to enable new algorithms or architectures
  that reduce memory usage.

We instead seek a generic and automatic solution to the problem
  of reducing program memory consumption.
By generic, we mean a solution
  that is applicable to arbitrary programs
  in a given programming language.
By automatic, we mean a solution
  that requires no programmer annotations
  or other modifications to the program itself.
In other words, we seek a runtime system
  that would automatically reduce the memory consumption
  of a program.
Moreover, we would like to reduce memory consumption
  below the level of live heap memory
  (unlike garbage collection)
  that adapts to fine-grained program behavior
  (unlike swapping.)

\name is a generic and automatic runtime system
  that reduces program memory consumption
  by evicting values from memory
  and recomputing them when necessary.
That is, \name is a runtime
  for a simple purely-functional program language
  (which we call $\lambda_Z$)
  that pairs every value on the heap
  with the information necessary
  to recompute that value if it were evicted.
\name can then evict any heap value
  at any point in time,
  thereby reducing program memory consumption,
  working below the level of live heap memory
  and adapting to fine-grained program behavior.
In doing so,
  \name preserves the original program semantics,
  makes forward execution progress
  and is relatively efficient in its use of memory.
Of course, the cost of replay
  is longer program runtime
  (since individual program steps may be replayed many times);
  in this way, \name introduces a space-time tradeoff,
  where programs run faster when more memory is available.
That said, various optimizations,
  such as batching, a fast path for hot objects,
  and union find for fast eviction,
  allows \name to run at moderate overhead
  compared to a more traditional runtime.

Moreover, \name achieves its memory reduction
  without adding a commensurate memory overhead.
That is, the amount of metadata stored by \name
  to enable recomputation
  is proportional to the total memory,
  and independent of how long the program has been running.
To achieve this,
  \name stores replay points sparsely,
  so that replaying a given computation
  might involve replaying earlier computations as well,
  starting from whatever earlier replay point is available.
The sparse replay points
  allow \name to keep the total memory dedicated to replay limited
  while still enabling it to recompute any arbitrary value.

To evaluate \name,
  we test 10 $\lambda_Z$ programs
  that implement common algorithmic tasks
  such as balanced binary trees, list and array manipulation,
  and recursive functions.
Each program is run with different memory limits,
  ranging as low as \todo{one thousandth}
  of a traditional run-time's memory consumption.
\name allows the programs to run
  despite severe memory restrictions,
  with a $10\times$ reduction in memory
  typically resulting in a $5\times$ increase in running time.

In short, this paper contributions:
\begin{itemize}
\item Zombie, a runtime that saves memory
  for arbitrary programs without programmer effort;
\item Proofs that Zombie
  preserves program semantics, makes progress, and is efficient;
\item A collection of optimizations
  that enable Zombie to run with moderate runtime overhead.
\end{itemize}

\section{Overview}	
The tock tree serve as a cache \todo{insert this sentence somewhere}

Section: The CEKR Machine (???)
Replay Stack
the section with lots of greeks
argue about progress

Section: Implementation (3pg long)
Heuristic
Loop Unrolling

Key question: How to get the replay stack small?
Key question: Garbage Collection/Eager Eviction

Summarize the meeting into key step

Double O(1)
	\section{Core Language}	

Zombie works on a purely functional language with products, sum types,
and first-class functions called $\lambda_Z$. For simplicity in this
paper, we treat the language as untyped. The syntax of $\lambda_Z$ is
shown in \Cref{fig:syntax}; its semantics are standard. The full
Zombie implementation supports additional features, such as primitive
types and input/output; \Cref{sec:impl} describes how these features
are layered on top of the core implementation described here.

Importantly, because Zombie is purely functional, programs are totally
deterministic, in the sense that evaluating a given expression in a
given environment always returns the same result. This is essential
for Zombie to work correctly.

\subsection{CEK Machine}

The key insight of Zombie is to assign an unique identifier to any
value ever allocated during the program's execution. Conceptually, it
identifies each value with the execution step that allocated it. It
does this using a variant of the CEK machine.

The CEK machine is a well-known abstract machine for executing untyped
lambda calculi, where the machine state consists of three parts:

\begin{enumerate}
	\item \textcolor{blue}{C}ontrol, the expression currently being evaluated.
	\item \textcolor{blue}{E}nvironment,
          a map from the free variables of the Control to their values.
	\item \textcolor{blue}{K}ontinuation, which is to be invoked
          with the value the Control evaluates to.
\end{enumerate}

\todo{We should use $C$ for expressions, to match the CEK terminology.}

In our variation of the CEK machine, we split the evaluation phase
from the invocation of the continuation, resulting in a machine state
that looks like this:
\[
\text{State} \operatorname{::=} \Eval~E~\Env~K \mid \Apply~K~V
\]
In other words, our CEK machine can be in an $\Eval$ or an $\Apply$
state; the $\Eval$ stores the classic control, environment, and
kontinuation, while the $\Apply$ state stores just a continuation and
a value. The precise syntax of values and kontinuations are given in
\Cref{fig:defs}.

Execution in the CEK machine involves a series of transitions between
these machine states; in other words, it is a transition system. To
run a program $C$ in the CEK machine, one first sets up an initial
state $(\Eval~C~\{\}~\Done)$ whose Control is the expression to
evaluate, whose Environment is empty, and whose Kontinuation is a
special $\Done$ continuation. The rules of the CEK machine are then
used to transition from one machine state to another, until finally
reaching the state $\Apply~\Done~V$. In that state, $V$ is the result
of evaluating of $C$.%
\footnote{Note that, because $\lambda_Z$ is untyped, it contains
non-terminating programs using, for example, the $Y$ combinator. In
the CEK machine, these programs create an infinite sequence of machine
states that do not include a terminating $\Apply~\Done~V$.}

A notable property of the CEK machine is that each transition performs
a bounded amount of work. Contrast this to a traditional operation
semantics, where the semantics of a \sCase statement might be something
like:

\begin{mathpar}
 \inferrule{
   \Gamma \vdash e \to^* \sLeft~\Gamma' \vdash V
 }{\Gamma \vdash \sCase~e~x~e_l~y~e_r
   \to \Gamma', x : V \vdash e_l
 }
\end{mathpar}

Here the antecedent of the inference rule might perform arbitrarily
many steps, and thus an abitrary amount of computation; derivations
thus form a tree. In the CEK machine, no such steps exist, and
derivations in the CEK machine form a flat list. This property of the
CEK machine is illustrated graphically in \Cref{fig:constant}.

\begin{figure}
\includegraphics[width=0.5\columnwidth]{1}
\caption{the machine does a small constant amount of pointer lookup}
\label{fig:constant}
\end{figure}

\subsection{Determinism and Tocks}

Importantly, the CEK machine is linear and deterministic. This means
that every CEK machine state transitions to at most one state. That in
turn means that if we were to ``rewind'' a CEK machine, putting it in
an earlier state, it would transition through the exact same sequence
of states in the exact same order. This determinism or
``replayability'' is essential for Zombie to work, and is illustrated
graphically in \Cref{fig:replayability}.

\begin{figure}
\includegraphics[width=0.5\columnwidth]{0}
\caption{the deterministic, linear nature of the CEK machine}
\label{fig:replayability}
\end{figure}

One key property guaranteed by determinism is that, for a given
initial program $C$, the machine state at any point during $C$'s
execution can be uniquely identified by how many steps have been
executed since the initial state. In other words, the initial state is
identified with the number $0$, the next state it transitions to with
the number $1$, and so on. This state number, which we call a
``tock'', is logically unbounded, but in our implementation it is
stored as a 64-bit integer. In our implementation, that suffices for
several decades of runtime on current hardware. 

\subsection{Heap Memory}

Because we are interested the total memory usage of $\lambda_Z$
program, our variant of the CEK machine includes an explicit heap and
explicit pointers. That is, in our formalization values $V$ are
formalized as pointers $P\langle \VCell \rangle$ to ``value cells''.
Values cells---which can be closures, products, and sums---in turn
contain values, that is, pointers.

During execution, the CEK machine looks up these values using an
explicit heap $H$. To interact with the heap, the CEK machine uses two
functions. $\Lookup : (P \langle X \rangle, H) \to X$ dereferences a
pointer in the heap. $\Alloc : (X, H) \to (P \langle X \rangle, H')$
stores a value in the heap, returning a pointer to it and the updated
heap. Transitions in the CEK machine call $\Lookup$ and $\Alloc$ in
order to make $\Eval$ and $\Apply$ steps, as shown in \Cref{fig:eval}
and \Cref{fig:apply}.

Importantly, every CEK machine step (whether $\Eval$ or $\Apply$)
makes at most two calls to $\Lookup$ and $\Alloc$, of which at most
one is an $\Alloc$ call. This means that each allocation the program
makes can be uniquely identified by the tock for the machine state
where it is allocated. This identification is the core abstraction
that drives Zombie's implementation.

\newcommand{\mytableshape}{p{6em} p{2.6em} p{1em} p{0.45\textwidth}}
\begin{figure}
	\begin{tabular}{\mytableshape}
		Name & $N$ & $::=$ & A set of distinct names \\
		Expr & $E$ & $::=$ & $
		N \mid
		\sLet~N~E~E \mid
		\sLam~N~E \mid
		\sApp~E~E \mid
		\sProd~E~E \mid
		\sZro~E \mid
		\sFst~E \mid
		\sLeft~E \mid
		\sRight~E \mid
		\sCase~E~N~E~N~E $
	\end{tabular}
	\caption{The syntax of $\lambda_Z$.}
        \label{fig:syntax}
\end{figure}

\begin{figure}
	\begin{tabular}{\mytableshape}
		Continuation & $K$ & $::=$ & $P \langle \KCell \rangle$ \\
		
		KCell & & $::=$ & $
		\Done \mid
		\KLet~N~\Env~E~K \mid
		\KApp_0~\Env~E~K \mid
		\KApp_1~\Env~N~E~K \mid
		\KProd_0~\Env~E~K \mid
		\KProd_1~V~K \mid
		\KZro~K \mid
		\KFst~K \mid
		\KLeft~K \mid
		\KRight~K \mid
		\KCase~\Env~N~E~N~E~K $ \\
		
		Value & $V$ & $::=$ & $P\langle \VCell \rangle$ \\
		VCell & & $::=$ & $
		\Clos~\Env~N~E \mid
		\VProd~V~V \mid
		\VLeft~V \mid
		\VRight~V $ \\
		
		Environment & $\Env$ & ::= & $(N, V) \dots$ \\
		State & & ::= & $\Eval~E~\Env~K \mid \Apply~K~V $ \\
	\end{tabular}
	\caption{Definitions for the CEK Machine}
        \label{fig:defs}
\end{figure}

\begin{figure}
	\begin{mathpar}
		\inferrule{ }{\text{State}, H \leadsto \text{State}, H} \and
		\inferrule{ }{\Eval(N, \Env, K), H \leadsto \Apply(K, \Env(N)), H} \and
		\inferrule{\Alloc(\KLeft~K, H) = (P, H')}{\Eval(\sLeft~X, \Env, K), H \leadsto \Eval(X, \Env, P), H'} \and
		\inferrule{\Alloc(\KRight~K, H) = (P, H')}{\Eval(\sRight~X, \Env, K), H \leadsto \Eval(X, \Env, P), H'} \and
		\inferrule{\Alloc(\KProd_0~K~R, H) = (P, H')}{\Eval(\sProd~L~R, \Env, K), H \leadsto \Eval(L, \Env, P), H'} \and
		\inferrule{\Alloc(\KZro~K, H) = (P, H')}{\Eval(\sZro~X, \Env, K), H \leadsto \Eval(X, \Env, P), H'} \and
		\inferrule{\Alloc(\KFst~K, H) = (P, H')}{\Eval(\sFst~X, \Env, K), H \leadsto \Eval(X, \Env, P), H'} \and
		\inferrule{\Alloc(\KCase~\mathit{LN}~L~\mathit{RN}~R~\Env, H) = (P, H')}{\Eval(\sCase~X~\mathit{LN}~L~\mathit{RN}~R, \Env, K), H \leadsto \Eval(X, \Env, P), H'} \and
		\inferrule{\Alloc(\KLet~A~K~C~\Env, H) = (P, H')}{\Eval(\sLet~A~B~C, \Env, K), H \leadsto \Eval(B, \Env, P), H'} \and
		\inferrule{\Alloc(\KApp_0~K~X, H) = (P, H')}{\Eval(\sApp~F~X, \Env, K), H \leadsto \Eval(F, P), H'} \and
		\inferrule{\Alloc(\Clos~\Env(\text{fv})\cdots N~E, H) = (P, H')}{\Eval(\sLam~N~E, \Env, K), H \leadsto \Apply(K, P), H'}
	\end{mathpar}
	\caption{Abstract Machine Transition: Eval}
        \label{fig:eval}
\end{figure}

\begin{figure}
	\begin{mathpar}
		\inferrule{\Lookup(P, H) = \KLeft~K \and \Alloc(\VLeft~V, H) = (P', H')}{\Apply(P, V), H \leadsto \Apply(K, P'), H'} \and
		\inferrule{\Lookup(P, H) = \KRight~K \and \Alloc(\VRight~V, H) = (P', H')}{\Apply(P, V), H \leadsto \Apply(K, P'), H'} \and
		\inferrule{\Lookup(P, H) = \KCase~\Env~\mathit{LN}~L~\mathit{RN}~R~K \and \Lookup(V, H) = \VLeft~V}{\Apply(P, V), H \leadsto \Eval(L, \Env(\mathit{LN} := V), K), H} \and
		\inferrule{\Lookup(P, H) = \KCase~\Env~\mathit{LN}~L~\mathit{RN}~R~K \and \Lookup(V, H) = \VRight~V}{\Apply(P, V), H \leadsto \Eval(R, \Env(\mathit{RN} := V), K), H} \and
		\inferrule{\Lookup(P, H) = \text{KProd0}~\Env~R~K \and \Alloc(\KProd_1~V~\Env~K, H) = (P', H')}{\Apply(P, V), H \leadsto \Apply(K, P'), H'} \and
		\inferrule{\Lookup(P, H) = \KProd_1~L~K \and \Alloc(\VProd~L~V, H) = (P, H')}{\Apply(P', V), H \leadsto \Apply(K, P'), H'} \and
		\inferrule{\Lookup(P, H) = \KZro~K \and \Lookup(V, H) = (\VProd~X~Y)}{\Apply(P, V), H \leadsto \Apply(K, X), H'} \and
		\inferrule{\Lookup(P, H) = \KFst~K \and \Lookup(V, H) = (\VProd~X~Y)}{\Apply(P, V), H \leadsto \Apply(K, Y), H'} \and
		\inferrule{\Lookup(P, H) = \KLet~A~\Env~C~K}{\Apply(P, V), H \leadsto \Eval(C, \Env[A := V], K), H'} \and
		\inferrule{\Lookup(P, H) = \KApp_0~\Env~X~K \and \Lookup(V, H) = (\Clos~\Env'~N~E, H) \and \Alloc(\KApp_1~\Env'~N~E~K, H) = (P', H')}{\Apply(P, V), H \leadsto \Eval(X, \Env, P'), H'} \and
		\inferrule{\Lookup(P, H) = \KApp_1~\Env~N~E~K}{\Apply(P, V), H \leadsto \Eval(E, \Env[N := V], K), H'}
	\end{mathpar}
	\caption{Abstract Machine Transition: Apply}
        \label{fig:apply}
\end{figure}

\begin{figure}
	\begin{tabular}{\mytableshape}
		Heap & $H$ & $::=$ & An abstract key value store \\
		Pointer$\langle X \rangle$ & $P\langle X \rangle$ & $::=$ & Key into heap with value type $X$ \\
		Lookup & & : & $(\text{Pointer}\langle X \rangle, H) \to X$ \\
		Alloc & & : & $(X, H) \to (\text{Pointer}\langle X \rangle, H)$ \\
	\end{tabular}
	\caption{Heap API}
        \label{fig:heap}
\end{figure}

	\section{Uncomputing and Recomputing}
\subsection{Tock}
The key insight in zombie is to introduce an abstraction layer on pointers and on the heap, turning pointers, keys into memory space, into tocks, which is 64bit int, keys into logical time.

Tock start at 0, and is increased by 1 on every transition step in the abstract machine, or on every allocation. This created a one-to-one mapping between each transition/lookup(event), with each tock less then the current tock. This had unveil the linear ordering between each events. In particular, a Value allocated during tock X can be recreated by re-executing a transition starting at tock Y < X, because the CEK machine is deterministic, and will replay faithfully what had happened.

Note that for any tock X value, there are, barring edge cases many tock Y < X transition that can replay said value. conversely, for any tock Y, there are multiple tock X > Y value it replay. The one-to-many mapping between transition and values permit us to store asymptotically less metadata than the amount of allocated values.
\subsection{Tock Tree}
To exploit this linearity, the memory space is transformed into a global binary search tree, a Tock Tree, keyed by the tocks.

Each node in the Tock Tree correspond to a state transition. A node at tock t, representing the transition starting at tock t, contain an array of cells(value representations) it allocated, at tock t+1, t+2\dots, paired with the state it transit to. Note that it store the transit-to state, but not the transit-from state, for that state is useless: the transit-from state can recompute the allocated cells during the transition, but we had already stored it anyway!
\begin{figure}
	\includegraphics[width=0.5\columnwidth]{2}
	\caption{a node in the tock tree}
\end{figure}
\begin{figure}
	\includegraphics[width=0.5\columnwidth]{3}
	\caption{the tock tree with multiple nodes}
\end{figure}

Unlike binary search tree, which fail when a lookup does not find the precise match, the tock tree is lenient. On a lookup of key x, if said key is not in the tock tree, it will return the node with the largest key y < x instead. Intuitively, this represent returning the latest transition that can replay to the queried node.
\begin{figure}
	\includegraphics[width=0.5\columnwidth]{4}
	\caption{lookup failure return the latest earlier node}
\end{figure}

This allow us to remove any non-leftmost node from the tock tree. After the removal, the query that originally return the removed node, will return the node slightly earlier then that, which we can then replay to regenerate the removed node. In fact, this is the implementation of uncomputation in our system, and any non-leftmost node can be removed, to save memory at any given time.
\subsection{Happy path}
Every state transition and pointer lookup is then converted into node insertion and node lookup into this data structure. After we had completed a transition, we construct the node, packing the allocated cells during the transition and the transit-to state, inserting the node onto the tock tree.

Originally, transition might require pointers lookup. Such lookup is translated to a query to the tock tree with the given node. Ideally speaking, the returned node contain the cell that correspond to the given tock. We can then convert the tock into the corresponding cell and continue execution. Note that we still need the tock tree to be lenient in this case, as the key of the node denote the beginning of the transition, not the cell itself.
\subsection{Sad Path}
Sadly, as we had removed nodes from the Tock Tree, we might not be able to retrieve a cell directly, and might need to recompute it - the point of the paper.

To recompute a value at tock t, the following steps are taken:
\begin{enumerate}
	\item Suspend the current execution into another kind of continuation, Replay Continuation (RK).
	\item Execute the transit-to state from the lookuped node in the tock tree, until tock reach t.
	\item Resume RK with the cell created at tock t.
\end{enumerate}
Note that the Replay Continuation operate at a more fine-grain granularity then that of the normal continuation. This is because a single transit step might do multiple lookup, but RK need to correspond to a lookup failure in such a transition. Otherwise the requested value might be immediately uncomputed again, and the whole process enter an infinite loop.

After the above 3 steps, the execution shall continue as if no replay had happens at all, and lookup return a node with the cell we wanted. In other words, the happy path and the sad path should converge.

During the replaying process, more lookup might be issued, and those lookup might need more replay - replay is recursive. Just like the classical continuation at the CEK machine, the Replay Continuation need to be recursive, and form a stack as well.

\subsection{Reified Continuation}
We treat continuations also as values, and label them with tock/put them in the tock tree, just like any other values. This allow us to also uncompute continuation as well.
	\section{Implementation}
We implemented the zombie runtime as a C++ library, alongside a C++ compiler that links to and calls the library.
\subsection{Tock Tree}
To exploit the temporal/spatial locality, and the 20-80 law of data access (cite?), the tock tree is implemented as a slight modification of a splay tree.

This design grants frequently accessed data faster access time. Crucially, consecutive insertion takes amortized constant time.

The tock tree is then modified such that each node contains an additional parent and child pointer. The pointers form a list, which maintains a sorted representation of the tock tree. On a query, the tock tree does a binary search to find the innermost node, then follows the parent pointer if that node is greater than the key. This process is not recursive: the parent pointer is guaranteed to have a smaller node than the input key, as the binary search will yield the exact value, or the largest value less than the input, or the smallest value greater than the input.
\subsection{IO}
While our language is purely functional, we allow arbitrary IO effect directly. As IO is not referentially transparent, such effect cannot be replayed safely, and any node that executed a IO effect cannot be evicted, and therefore should not be considered for eviction.
\subsection{Language Implementation}
For implementation simplicity and interoperability with other programs, zombie is implemented as a C++ library, and the Cells are ref-counted. Our evaluation compiles the program from the applicative programming language formalized above(give name), to C++ code. The program is linked with mimalloc, with malloc and free being hooked to track the current memory consumption. Between every state transition, we check if the memory consumption is above a given memory limit. If so, we evict from the tock tree until it is no longer the case.
\subsection{Optimization}
\subsubsection{Fast access path}
Querying the tock tree for every value is slow, as it requires multiple pointer traversal.
To combat this, each Value is a Tock paired with a weak reference, serving as a cache, to the Cell. When reading the value, if the weak reference is ok, the value is returned immediately. Otherwise, the default path is executed, and the weak reference is updated to point to the new Result.
\subsubsection{Loop Unrolling}
Each node in the tock tree is quite heavy. It contain multiple pointers (from the tock tree data structure), with the next state, which is a function pointers with a list of input tocks. As each node only store 0-1 objects, this pair each object with a high overhead. To combat this, and to avoid frequent insertion of node objects into tock tree, multiple state transition is packed into one. This mean that each node now store a vector of cells, with the next state unchanged. In our implementation, each node store 32 objects.
\subsection{Picking Uncomputation Candidate}
\subsubsection{The Policy}
In principle, we want to find the node that is the quickest to replay, take the most space, and is the stalest, to evict. More specifically, we want to find the node with the highest score = SpaceConsumption * TimeSinceLastAccess / ComputeTime.

To achieve this, for each node, we measure and store it's space consumption, time taken to execute, and last access time. Whenever a object is accessed, we update the corresponding node's last access time accordingly.
\subsubsection{GD}
As doing a linear search to find the node with the highest score is computationally infeasible, requiring a linear amount of work to free up a constant amount of memory, instead we adopt the greedy dual technique in caching.

Greedy dual is a technique in caching that approximate a eviction policy of X * staleness for arbitrary constant X.
It's implementation consists of a priority heap alongside a value L.
L represent roughly, the total amount of 'compute' that had been evicted.
L start at 0, increasing by X whenever a value is evicted, and whenever an object is inserted into the cache, it have the priority of X + L.
\subsubsection{Union Find}
We want to avoid long, recursive replay, and instead prefer replay to complete as soon as possible. This suggest that our compute cost should be recursive: if replaying to node A require replaying node B, A's cost should also include B's cost. As tracking this require huge computational resource, we instead use a union find data structure to approximate recursive cost. On a high level, when node A and node B is evicted, and A depend on B, A and B should be in the same equivalent class.
	\section{Formal Guarantee}
In this section we prove that CEKR preserve(step to same value as CEK), progress(always step when CEK can step), and efficient. 
\subsection{Correctness}
To prove that CEKR is correct (progress and preservation), we first define a transition on CEKR, GhostStepping. GhostStepping transit by assuming lookup always return the matching value, even if the value is evicted. Essentially, GhostStepping allow us to connect the CEKR machine with the CEK machine. 

We then define another transition, OverStepping. A single step in OverStepping ignore all replay requested by lookup failure, continuing execution until the replays issued by the lookups had been resolved and returned. Our main theorem is a correspondence between GhostStepping and OverStepping, both of which happens on the CEKR Machine.

\subsubsection{Tock Tree Setup}
To reason about the CEKR machine, we needed to reason about nodes in the tock tree, even if such node is evicted.

While we can reify the tock tree insertions into a list, forming a trace semantic and reasoning on top of the list, it is easier to allow querying the tock tree for evicted data. In our formalization, we assume that the tock tree does not really evict; Instead, each node is now paired with a boolean bit, indicating whether it has been evicted or not. We will then have a function QueryGhost : (TockTree, Tock) -> (Bool, Node) that returns the Node, along with a boolean indicating whether the node is present.

In some cases, we do not care about the eviction status of the tock tree, and we only care about the Node corresponding to each tock. For this purpose, we define two tock tree X, Y, to be 'equal modulo eviction', if they contains the same node, under different eviction status. Formally speaking, X eqmodev Y $\iff$ $\forall$ t, $\exists$ xev yev node LookupGhost(t, X) = (xev, node) and LookupGhost(t, Y) = (yev, node).

Using QueryGhost, we can define a function, LookupGhost, that return the Cell corresponding to a tock. We can then use LookupGhost to connect the CEKR machine to the CEK machine. With LookupGhost, we can connect the value from the CEKR machine to the CEK machine. More specifically, we can now define a Sem(standing for semantic) function, that converts a CEKR-value to a CEK-value under a given tock tree. The function works by recursively calling LookupGhost.

\begin{figure}
	\begin{mathpar}
		\inferrule{ }{ LookupGhost : (Tock, TockTree) \to CEK-Value | CEK-Continuation } \and
		\inferrule{ QueryGhost(t, tt) = (\_, Node(Just x, st)) }{LookupGhost(t, tt) = x } \and
		\inferrule{ }{Sem : (CEKR-Value|CEKR-Continuation, TockTree) \to CEK-Value | CEK-Continuation} \and
		\inferrule{ LookupGhost(t, tt) = VProd(l, r) }{Sem(t, tt) = VProd(Sem(l, tt), Sem(r, tt))} \and
		\inferrule{ }{Convert : CEKR-State, TockTree \to CEK-State} \and
		\inferrule{ }{Convert(Eval(C, Env, K, t), tt) = Eval(C, Env[x = Sem(x, tt)...], Sem(K, tt))} \and
		\inferrule{ }{Convert(Apply(K, cell, t), tt) = Apply(Sem(K, tt), Sem(cell, tt))}
	\end{mathpar}
\end{figure}
\subsubsection{Ghost Stepping}

Alongside QueryGhost, is the idea of GhostProduce and GhostStepping \todo{this name suck}. GhostProduce is a function from (CEKR-State, TockTree) tuple to (Node, TockTree). GhostProduct represent executing a single transition via QueryGhost (so it need not involve any replay), returning the produced Node alongside the new TockTree, with the Node inserted. GhostProduct allow us to define GhostStepping, which is a transition from CEKR-Machine to CEKR-Machine, mimicing a CEKR-transition that cannot fail, as it use QueryGhost instead of Query. GhostStepping is unimplementable, but just like our formalization of Tock Tree and QueryGhost, it is purely for formalization purposes. As GhostStep does not need any replay, it does not have the replay continuation. GhostStep serves as a bridge between CEK and CEKR, connecting the two semantics. Once this is completed, we can only talk about correspondence between GhostStep and normal step, inside CEKR, without mentioning the CEK machine at all.

Lemma: CEK stepping is deterministic: $\forall$ CEK-State X Y Z, X $\leadsto$ Y and X $\leadsto$ Z $\implies$ X = Y.
proof: trivial.

Lemma: Ghost stepping is deterministic: $\forall$ CEKR-Machine X Y Z, X $\leadsto$ Y and X $\leadsto$ Z $\implies$ X = Y.
proof: trivial.

Lemma: Ghost stepping is congruent under eqmodev: $\forall$ (State | Return Cell) S S', ReplayContinuation RK RK', TockTree x y x', eqmodev x y and (S, RK), x $\leadsto$ (S', RK'), x' $\implies$ $\exists$ y', (S, RK), y $\leadsto$ (S', RK'), y' and eqmodev x' y'.
proof: ghost stepping does not read from eviction status.

\subsubsection{Well Foundness}
A tock tree T is well-founded if:
\begin{enumerate}
    \item The leftmost node exists and is not evicted. (root-keeping)
    \item Forall node N inserted at tock T, N only refers to tock < T. (tock-ordering)
    \item Forall node pair L R (R come right after L in the tock tree), (L.state, T) GhostProduce R.
    \item Additionally, a (state, tock tree) pair is well-founded if the state is in the tock tree. (state-recorded)
\end{enumerate}

Lemma: ghost stepping preserves well-foundedness.
proof:
\begin{enumerate}
    \item root-keeping is maintained by the tock tree implementation and is a requirement.
    \item tock-ordering is maintained - check each insert.
    \item replay-correct is maintained: the transit-from state must be on the tock tree due to state-recorded. If it is not the rightmost case, due to replay-correct we had repeatedly inserted a node. Since ghost-stepping is congruent under eqmodev replay-correct is maintained. If it is the rightmost state, a new node is inserted. All pairs except the last pair are irrelevant to the insertion because due to tock-ordering they cannot refer to it. For the last pair, the ghost-step must reach the same state due to determinism in ghost-stepping.
    \item state-recorded is maintained: we just inserted said state.
\end{enumerate}

\subsubsection{CEK and the CEKR}
Theorem: under well-foundness, CEK-step and ghost-stepping preserve equivalence.

Formally: given CEK State X X', CEKR Machine ((s, NoReplay), tt), Convert(s, tt) = X and T is well-founded, X $\leadsto$ X', then: ((s, NoReplay), tt) $\leadsto$ ((s', NoReplay), tt'), such that Convert(s', tt') = X'.

\todo{is this theorem actually needed? or do we need a version with arbitrary replaycontinuation, which allow return as well? formalizing that is a bit hard}

proof: wellfoundness ensures we do not write to the old state and replace the old node with other values. other part of the proof is trivial.

The above theorem establishes a connection between the CEK machine and the CEKR machine, by ignoring the replaying aspect of the CEKR machine, and only using it as if all values are stored in the tock tree (even though some nodes might already be evicted!). We will next connect ghost-stepping with regular stepping in the CEKR machine.

\subsubsection{Over-Stepping}
Our normal CEKR transition, however, does not one-to-one preserve with CEK transition, as it also include transition that failed and request Replay. To overcome this, we introduced Over-Stepping \todo{name suck}, which take a step ignoring replay and what happens inside replay, in resemblance to Step-Over in debugger. Formally speaking, ((S, rk), tt) step to ((S', rk'), tt') if len(rk) >= len(rk'), ((S, rk), tt) step-star to ((S'', rk''), tt''), with all len(rk'') > len(rk), and ((S'', rk''), tt'') step to ((S', rk'), tt'). 

Theorem(Preservation): Under Well Founded Tock Tree, GhostStepping and OverStepping preserve.

Formally: if tt is WellFounded, ((S, rk), tt) GhostStep to ((S', rk'), tt'), and ((S, rk), tt) OverStep to ((S'', rk''), tt''), then S' = S'', rk' = rk'', and tt' eqmodev tt''.
Proof: We induct on the number of transition OverStep actually make. It cannot be 0, so the base case start at 1.
For the base case, there can be no replay, which mean the query must succeed, and GhostStepping match this single CEKR-step.
For the inductive case, where there are n+1 transition, the query must fail (otherwise the transition amount is 1, and it is the base case), so rk become rk\_ = (Replaying t rh rk). ignoring the initial failing transition, the rest of the transition is a sequence of Over-Stepping that start with rk\_, with the ending one decreasing rk\_ back into rk, where the rest maintain rk\_. With our inductive hypothesis, this sequence of Over-Stepping can be seen as a sequence of Ghost-Stepping. Then, by the well-foundness of the tock tree, this query-failure, followed by ghost-stepping preserving rk\_, followed by a return ghost-stepping, is equivalent to a lookup that had not failed, so a single ghost stepping.

Theorem(Progress): Under Well Founded Tock Tree, OverStepping is not stuck if GhostStepping is not stuck.

Formally, if tt is WellFounded, ((S, rk), tt) OverStepping to M if ((S, rk), tt) GhostStep to M'.

Note that our Machine can get stuck as we are untyped. To overcome this, we piggyback the concept of progress on GhostStepping. Maybe our Preservation and Progress proof should be merged as one.

Proof: 

If our query succeed, our theorem trivially hold, as OverStepping only consist of a single CEKR-step.
Let's consider the case where it failed and request a replay to tock t. Let's called this ReplayContinuation we transit to rk. If overstepping is stuck, either there is infinitely many CEKR step with ending ReplayContinuation len >= len(rk), or CEKR is stuck during replay to tock t. But that CEKR-step had been played before, and our tock tree is well founded, so it is impossible. The only case that remain is the infinitely many CEKR step.

In such case, for each CEKR-Machine in the infinitely long transition, extract out all the request tock from the ReplayContinuation, forming a list of increasing tocks, and with the head of the list being the tock from the CEKR-State.

We can define a ordering on this list of tocks:
\begin{enumerate}
	\item t:ts < ts
	\item a > b $\to$ a:ts < b:ts
	\item x < y $\to$ t:x < t:y
	\item a < b \and b < c $\to$ a < c
\end{enumerate}
It is clear that this relation is transitive and antisymmetric. Additionally for any tock list ending with X, there are a finite amount of smaller elements ending with X, as the list must contain an increasing number of tocks.

However, each CEKR step in the infinite sequence must decrease on this list, and must also end with X (otherwise overstepping can produce and is not stuck), so we had reached a contradiction.

\subsection{Performance}
\subsubsection{Pebble Game}
Let $G=(V,E)$ be a directed acyclic graph with a maximum in-degree of $k$, where $V$ is a set of vertices and $E$ is a set of ordered pairs $(v_1, v_2)$ of vertices.

For each $v\in V$, we can
\begin{itemize}
    \item place a pebble on it if $\forall u\in V$ that $(u, v)\in E$, $u$ has a pebble, which means all predecessors of $v$ have pebbles.
    \item remove the pebble on it if $v$ already has a pebble.
\end{itemize}

The goal is to place a pebble on a specific vertex and minimize the number of pebbles simultaneously on the graph during the steps.
\subsubsection{Converting CEKR Program into Pebble Game}
\subsubsection{Theorem}
Theorem: for a program that execute for N step, there exists a eviction strategy such that it consume at most O(n / log(n)) memory.
Proof: 
	\section{Evaluation}
	%% \subsection{Motivation}
% Program execution consume both space and time. While there are tons of research focused on reducing the time spent running a program, memory usage reduction had been consistently underappreciated.

% Yet, saving memory is still an important topic worth studying:
% \begin{itemize}
% 	\item Multi-tenancy. A end-user laptop execute multiple programs concurrently. As an example, a typical programmer might open an IDE to edit and compile code, Zoom for remote meeting, and additionally a browser with dozens of tabs open to lookup information on the internet. Among them, the worst is the web browser, as each tab is it's own separate process, with a renderer, a rule engine, along side a JavaScript runtime. All software above consume a decent amount of ram, and contend between themselves for memory.
% 	\item Huge input. While a typical image might have a resolution of around 1000 x 1000, large images might reach a size 100 to 1000 time larger than that. Images editors and viewers typically thrash or oom upon processing images of such size. Similarly, text editors typically process text of < 1MB, and fail on logs or other large text file reaching GBs. Likewise IDEs will parse and analyze the source code of a project, then stored the analyzed results for auto completion. Upon working on huge project such IDEs will crash.
% 	\item Intermediate state. A modern computer can read and write memory at a speed of GB per seconds. Without memory reclamation technique such as garbage collection, memory will run out at a matter of seconds. However, some applications is essentially out of reach from garbage collection, and must keep most if not all intermediate states around. These applications include time travelling debugger, jupyter notebook, reverse mode automatic differentiation for deep learning, incremental computation/hash consing, and algorithm that use search, such as model checker and chess bot.
% 	\item Low memory limit. Wasm have a memory limit of 2GB, and embedded device or poor people may have device with lower ram. GPU and other accelerator use their own separate memory and usually is of smaller capacity then main memory.
% \end{itemize}

% A typical and generic solution to reduce memory consumption is by uncomputation. Uncomputation trade space for time, by tagging values with the metadata that created it(thunk), and when memory is low, deallocate values which might be used later on, recomputing them back with the thunk when the value is needed again.

% Another generic solution is swapping, where upon low memory, pages are swapped to disk and swapped back when needed again. While swapping had been implemented for basically all operating systems, it is an especially bad solution for functional and object-oriented languages, as those languages allocate and deallocate lots of small object, yet swapping work on the granularity of pages, unable to separate out the hot objects, that had been recently accessed from the code objects, that had not been accessed.

% Typically, uncomputation is implemented in an ad-hoc, case by case basis. When a software consume more memory than desired, it's programmer can (1) look at the program, (2) decide which part is using excessive memory, then (3) add custom data type to record the thunk, alongside the code to replace it (4) implement a cache eviction policy to decide what data to evict(uncompute). Another solution is to analyze the algorithm carefully, replacing the algorithm with a specialized version with recomputation in mind, see gradient checkpointing/iterative deepinging depth first search/island algorithm.

% Needless to say, this process is extremely cumbersome, taking precious developers time away from more critical task such as optimizing cpu usage or implementing new features, so a generic, automatic approach is needed to reduce memory consumption.

% \subsection{Failed Attempt}
% On first glance, uncomputation is easy, as it's cousin, lazy evaluation, is heavily studied. In lazy computation, object might be represented as thunks, which, when needed, is then computed, and the thunk is replaced by the value. A solution to uncomputation is to modify thunk, so that, upon computing, the old function is still kept. This allow the thunk to uncompute as if it had never been recomputed.

% \begin{mathpar}
% 	$Lazy Evaluation: type 'a lazy = MkLazy of ('a, unit -> 'a) either ref \\
% 	Uncomputation: type 'a uncompute = MkUncompute of 'a option ref * (unit -> 'a)$
% \end{mathpar}

% Just like lazy evaluation, we can then uniformally lift all values on top of our modified value type, inserting code that force weak head normal form upon data access, and have some cache policy to decide what to uncompute, which merely rewrite the reference back into the empty optional value.

% However, there are multiple problems with this approach:
% \begin{enumerate}
% 	\item size measurement. We want to gather statistics to decide what values to evict, and one of the most important statistics is the memory a object consume. However, the heap form a complex object graph, and fast cardinality estimation is highly complex.
% 	\item dopple ganger. The translation will lift a list into nested uncompute. Now suppose variable X hold a list, while variable Y hold a sublist of X, sharing the same representation\todo{bad wording dont know how to fix}. When X is uncomputed and recomputed, the corresponding Y part will be duplicated, so there will be two representation of Y in memory. In the extreme case, such duplication can force the program to take exponentially more memory. 
% 	\item bread crumb. A thunk capture other value as free variable, which contain a thunk part, capturing more value. This recursive process capture all value that the current value transitively compute on. Since we still need memory to store the thunk itself, and since a thunk is typically larger then an object, anything we gained on uncomputation, will be offset by the metadata.
% \end{enumerate}

% While the above three problems seems difficult, we can observe they are all but mere manifestation of recursiveness. In particular, size measurement and dopple ganger deal with recursive objects, and bread crumb deal with recursive thunk. This indicate that a solution that handle recursiveness well naturally solve the three problems at once.

% \subsection{A recursive Solution}
% To combat recursiveness we take heavy inspiration from Abstracting Abstract Machine (AAM). AAM lift abstract interpretation onto programming languages with arbitrary feature. The critical problem there is likewise recursiveness, as a naive lifting might allow the lattice infinite merge, causing the analysis to non-terminate. AAM propose to use an abstract machine that abstract over pointers, allocations and lookups to manage recursiveness.

% Likewise, the three recursiveness problem listed above can be handled by a similar manner. We started from a purely functional language, specifying it's semantic with the CEK machine. We can then abstract over pointers/allocations/lookups similarly.

% Most importantly, our key insight is that by a careful handling of our core data structure, the tock tree, which will be explained later, we can remove a value, alongside the metadata(thunk) on that value, yet still recompute it later. This solve the breadcrumb problem which render the naive apporach unsalvageable.

% Alongside our solution is proof that it is both correct and efficient. Our correctness proof state that uncomputing will not effect the final result(preservation), and will not infinite loop(progress). What get computed by the raw, uncompute-free semantic, no matter what we evicted. Our efficiency proof state that if our memory consumption is O(n), where n is the amount of live objects: objects that take O(1) time to force into weak head normal form. In particular, this imply asymptotically we use no space to store any evicted object, yet we somehow can access the metadata that recompute them!

% We had implemented our solution, alongside a eviction policy that is both fast to compute, and make better decision over classical eviction policy like LRU, random, and GDSF.

\subsection{Motivation}
Program execution consumes both space and time. While there is tons of research focused on reducing the time spent running a program, memory usage reduction has been consistently underappreciated.

Yet, saving memory is still an important topic worth studying:
\begin{itemize}
    \item Multi-tenancy. An end-user laptop executes multiple programs concurrently. As an example, a typical programmer might open an IDE to edit and compile code, Zoom for remote meetings, and additionally a browser with dozens of tabs open to look up information on the internet. Among them, the worst is the web browser, as each tab is its own separate process, with a renderer, a rule engine, and a JavaScript runtime. All software above consume a decent amount of RAM and contend between themselves for memory.
    \item Huge input. While a typical image might have a resolution of around 1000 x 1000, large images might reach a size 100 to 1000 times larger than that. Images editors and viewers typically thrash or oom upon processing images of such size. Similarly, text editors typically process text of < 1MB, and fail on logs or other large text files reaching GBs. Likewise, IDEs will parse and analyze the source code of a project and then store the analyzed results for auto-completion. Upon working on a huge project such IDEs will crash.
    \item Intermediate state. A modern computer can read and write memory at a speed of GB per second. Without memory reclamation techniques such as garbage collection, memory will run out in a matter of seconds. However, some applications are essentially out of reach from garbage collection and must keep most if not all intermediate states around. These applications include time traveling debugger, Jupyter Notebook, reverse mode automatic differentiation for deep learning, incremental computation/hash consing, and algorithms that use search, such as model checker and chess bot.
    \item Low memory limit. Wasm has a memory limit of 2GB, and embedded devices or poor people may have devices with lower RAM. GPU and other accelerator use their own separate memory and usually are of smaller capacity than main memory.
\end{itemize}

At best, programmers today approach memory savings
  in an ad-hoc, case-by-case way:
  they examine their program
  and rewrite it to use less memory,
  possibly by using caching, lazy computation, or novel algorithms.
However, these case-by-case solutions
  are not always available,
  since they depend on the details of the computation.
Moreover, even when available, 
  such solutions can require substantial programmer effort,
  including rewriting the program data flow
  to enable new algorithms or architectures
  that reduce memory usage.

We instead seek a generic and automatic solution to the problem
  of reducing program memory consumption.
By generic, we mean a solution
  that is applicable to arbitrary programs
  in a given programming language.
By automatic, we mean a solution
  that requires no programmer annotations
  or other modifications to the program itself.
In other words, we seek a runtime system
  that would automatically reduce the memory consumption
  of a program.
Moreover, we would like to reduce memory consumption
  below the level of live heap memory
  (unlike garbage collection)
  that adapts to fine-grained program behavior
  (unlike swapping.)

\name is a generic and automatic runtime system
  that reduces program memory consumption
  by evicting values from memory
  and recomputing them when necessary.
That is, \name is a runtime
  for a simple purely-functional program language
  (which we call $\lambda_Z$)
  that pairs every value on the heap
  with the information necessary
  to recompute that value if it were evicted.
\name can then evict any heap value
  at any point in time,
  thereby reducing program memory consumption,
  working below the level of live heap memory
  and adapting to fine-grained program behavior.
In doing so,
  \name preserves the original program semantics,
  makes forward execution progress
  and is relatively efficient in its use of memory.
Of course, the cost of replay
  is longer program runtime
  (since individual program steps may be replayed many times);
  in this way, \name introduces a space-time tradeoff,
  where programs run faster when more memory is available.
That said, various optimizations,
  such as batching, a fast path for hot objects,
  and union find for fast eviction,
  allows \name to run at moderate overhead
  compared to a more traditional runtime.

Moreover, \name achieves its memory reduction
  without adding a commensurate memory overhead.
That is, the amount of metadata stored by \name
  to enable recomputation
  is proportional to the total memory,
  and independent of how long the program has been running.
To achieve this,
  \name stores replay points sparsely,
  so that replaying a given computation
  might involve replaying earlier computations as well,
  starting from whatever earlier replay point is available.
The sparse replay points
  allow \name to keep the total memory dedicated to replay limited
  while still enabling it to recompute any arbitrary value.

To evaluate \name,
  we test 10 $\lambda_Z$ programs
  that implement common algorithmic tasks
  such as balanced binary trees, list and array manipulation,
  and recursive functions.
Each program is run with different memory limits,
  ranging as low as \todo{one thousandth}
  of a traditional run-time's memory consumption.
\name allows the programs to run
  despite severe memory restrictions,
  with a $10\times$ reduction in memory
  typically resulting in a $5\times$ increase in running time.

In short, this paper contributions:
\begin{itemize}
\item Zombie, a runtime that saves memory
  for arbitrary programs without programmer effort;
\item Proofs that Zombie
  preserves program semantics, makes progress, and is efficient;
\item A collection of optimizations
  that enable Zombie to run with moderate runtime overhead.
\end{itemize}

	%\section{Uncomputation(5pg)}
\subsection{Motivation}
\subsection{Strawman}
\subsubsection{API}
\todo{unable to use the monadic api.}
\subsubsection{Semantic}
\subsection{Guarantee}
\subsection{Problems}
While the high-level idea seems straightforward, there are multiple subtle questions: 
\begin{itemize}
	\item Recursiveness. A value, for example, a list, might be recursive. Furthermore, there might be sharing inside such datastructure. In such a case, how do we measure the size of a value, which is a valuable statistics guiding what to uncompute?
	\item Partialness. Suppose a list is generated via anamorphism. We might want to uncompute the head of the list, but keeping some intermediate node inside the list.
	When recomputing said list, we need to be able to retrieve such intermediate node, to
	avoid spending extra time to recompute them, and extra memory to store them twice.
	\item Uncomputation candidates. Which value should we pick to uncompute?
	\item Breadcrumb. After a value is uncomputed, we need to store information needed to recompute said value. This become a bottleneck when most value is uncomputed, or when each value is small.
\end{itemize}

We present Zombie, a library for uncomptuation, alongside solutions to the questions above.
\subsubsection{recursiveness}
\subsubsection{doppelganger}
\subsubsection{breadcrumb}

	%\section{Plant vs Zombie(8pg)}
Combining the 3 insights above we now sketch an implementation of Zombie.
There is a global tock, which value begin from 0, increasing on every invocation of bind/return.
A Zombie<X> hold an Integer, tock, instead of value of type X.
The value is stored in the tock tree.
Whenever a call to return or bind is finished, we put a Node onto the tock tree.
The range of said node is the tock value at the begin and end of the invocation.

For return, the node contain the value.

For bind, the node contain the list of input Zombie, the output Zombie, and the function pointer.

In bind, we have to force values from Zombie X to X. This is done via looking it up from the tock tree. If the look up node is not a Value, but the Thunk, we need to rewind the tock to the beginning of said thunk, replay the value, get the value from the tock tree (todo: avoid infinite loop here), and restore the tock.

\section{Tock Tree API}
we need a data structure, such that when after removing a node, looking said node up does not return Nothing, but rather return a less precise answer.

Below is an API summing up what we need.

Tock = u64

Range = Tock * Tock

Make: TockTree X

Insert: TockTree X -> Range * X -> ()

Per two call to Insert, their range must either be nested or non-overlapping.

Lookup: TockTree X -> Tock -> Option (Range * X)

Remove: TockTree X -> Range -> ()

geometric series
\subsection{OLD}
All Zombies are threaded through a logical clock, a tock, stored as a u64.
During execution, the clock increases whenever makeZombie and bindZombie are called.

A Zombie contains, theoretically speaking, contains only the tock that the Zombie is created at.
To access the actual value of the Zombie, a ZombieNode, we rely on a global data structure.
It will be explained later.
(We also store a weakpointer to ZombieNode as a cache, to quicken access. This is just a cache, and is irrelevant to the rest of Zombie).
A ZombieNode holds X but is inherited from the Object class, which contain a virtual destructor and nothing else. This essentially allow one to type erase ZombieNode of different type, to put them into the same data structure.

We implement Zombie via two global data structures, a log and a pool.
The pool track all currently in-memory Zombie representation - it manage space
The log record actions taken used for recomputation - it manage time

The pool holds a vector of uniqueptr<Phantom>, the class that manage eviction.
It has api to evict objects, and api to score the profitability of eviction.
When we are low on memory, we can find a value in this vector, call evict(), and remove it.
This will free up memory in the log.

For every call to makeZombie and bindZombie, we keep track of when the function is called vs when the function exited. This establishes a tock range. 
For makeZombie, since it does not call any more functions, its interval length is 1, containing exactly the tock of the zombie.
Tock range overlap iff one contain another, iff one function calls another function. We then store tock range, paired with information on said function. 
For makeZombie we store a shared pointer to the ZombieNode. 
For bindZombie, we record the function used to compute it, as std::function<void(const std::vector<const void*>)>, alongside the tocks of all input Zombies, and the tock of output Zombie. We call it a Rematerializer. The input argument is type erased to void*, as Zombie is type polymorphic, but we need a type uniform interface here.

The log is some kind of interval tree. Each node in the tree stores its begin and end interval, alongside a value, and a BST from tock to sub nodes. The key of BST is the start of the sub node’s interval. When we query the log via a tock, we will find a value with the closest interval containing said tock. This mean, we might query for a Zombie, but instead of getting a ZombieNode, we get a Rematerializer.
In such a case, we replay the Rematerializer by fetching the ZombieNode from the tock (replaying recursively if necessary), setting the clock to begin of the function temporarily, and entering said function. Afterward we set the clock back, and fetch the value from a tree.
When we set the clock via rematerialization, we always set it to a smaller value, thus guaranteeing termination.

The log itself, when replaying, also serve as a memo table, indexed by tock. When we want to create a Zombie, we will skip when such Zombie already exist in the log. When we want to run BindZombie, we return the result tock when the precise Rematerializer exist in the log. This allow us to not compute the same function multiple time, if we know its result, and allow us to not store the same Zombie multiple time.

Note that all non-top-levels node in the tree can be removed. When we do so, we will free up memory, and when we need the value again, we can always replay a node higher up.

This design allow us to store n log n amount of metadata per n alive(nonevicted) objects.

Thus, when a Rematerializer dominate no value, and another value dominate Rematerializer, we remove it. (todo: implement). Idea: manage this using pool.

There is a slight error with the above design: maybe the value is recomputed and put on the tree, but then evicted, and fetch will not return a ZombieNode but a Rematerializer. To fix this we introduce a global data structure called Tardis, consist of a tock and a shared ptr<ZombieNode>*. When we create Zombie, when the tock match that of tardis, it will write to the shared ptr, holding the value longer.

\subsection{replaying semantic}
Program evaluate differently in replaying mode then in non-replay mode, as relevant Node might already be on the tock tree. When 

Every entering tock (lhs of a range) uniquely identify a return/bind.

% see https://www.khoury.northeastern.edu/home/wand/csg264/latex/mathpartir/mathpartir.pdf
\begin{mathpar}
	A-Formula \and
	Longer-Formula \and
	And \and The-Last-One
	
	\inferrule
	{aa \\\\ bb}
	{dd \\ ee \\ ff}
	
	\inferrule*
	{\inferrule* {aa \\ bb}{cc}
		\\ dd}
	{ee}
\end{mathpar}


\subsection{Tock Tree}
\subsection{Recomputation}
\subsection{Correctness Gurantee}
\subsubsection{Progress}
\subsubsection{Preservation}
\subsection{Asymptotic}

	%\section{Call Stack}
While the above construction can 
\subsection{Tail Call}
Call stack - get a node with inf right value onto tock tree, fix when 'actual' return

Clock stack - put the value on the node

C++ stack - trampoline

Replay stack - bounded by the tock tree size * intermediate node count, so it is linear w.r.t. tock tree.
\subsection{Non Tail Call}
By doing a continuation passing style transform all call become tail call.
The continuation might grow in size but we can zombified it.
	%\section{Cache Policy(2pg)}
note that the cache policy does not effect the correctness of the implementation, only the performance.

GDSF
\subsection{Measuring compute time}
measuring time is more complex then it seems.

need to be recursive and uncount recursive time.
\subsection{recursive cache policy}
\subsection{union find}
	%\section{Eval(3pg)}
\subsection{Small Program}
List Program

Pascal Triangle(2d, 3d)

Interpreter for LC

PE for LC

Interpreter for IMP

Compiler from IMP to LC

String Serialization and Deserialization

String Compression and Uncompression

Parsing and Unparsing

String = radix-tree/flat-array/linked-list

KD-tree, on shortest distance and on ray tracing

statistic \todo{ask}

Graph - DFS BFS MCTS

\todo{where does undo redo fit in?}
\subsection{Composite Program}
compose the programs to form larger program
	%\section{Case Study(2pg)}
	%\section{Related Work(2pg)}
\subsection{Memory-Constrained Algorithm}
Memory Limit is a common problem. Lots of different subfields of CS had experienced such problem, and developed specialized algorithm that trade space for time, by recomputing, for their own need. \todo{cite}. In particular, a famous algorithm, TREEVERSE, had been re-discovered independently multiple times in unrelated subfields.

Keeping uncomputation at each subfield not only cause multiple re-discovery and re-implementation, wasting valuable researcher/programmer time, but also come short in integration. Suppose a complex, memory-hungry software have 2 sub-parts, both of which are memory hungry. How much uncompute responsibility should each part take? What happens if two part depend on each other?
\subsection{Pebbling}
Space time tradeoff is a known-topic in Theoretical Computer Science. It is mostly modeled as a "pebble game" \todo{quickly explain}. It had been known that this problem is NP-complete and inapproximable. There are also generic result in this space. Hopcroft prove that for any program that take N time, it could be modified to take O(N / log(N)) space. However, the proof is non-constructive, and thus does not help building an automatic runtime like ours.

Our work mostly focus on the overhead problem - how to minimize the overhead of recording metadata needed for recomputation. How to make an API that is generic yet embeddable. In another words - our work is orthgonal to advances in pebble game. In fact, one could imagine replacing our cache policy with some of those algorithms.

We also provide a greedy cache policy that is both fast and make reasonably good decision on the list of benchmark, both real and synthetic, provided below, since the theoretical result is weak and non-constructive. There are some specialized result of pebble game on planar graph and on tree. One could imagine upon detecting planar graph/tree, switching from our cache policy to the theoretically optimal ones.
\subsection{Memoization}
Uncomputation is the dual to Memoization.
\subsection{Garbage Collection}
GC collect provably dead value, but we can also collect not-provably dead value, and even alive value.

It is like optimistic lock vs pessmistic lock.
\subsection{Compression}
Uncomputation is, fundamentally speaking, a form of compression.

Pigeon hole principle mean it is fundamentally impossible to have a generic compression algorithm. Uncomputation exploit the inductive bias that a program pointer is smaller then its memory consumption is larger.

	%\section{Constant Optimization}
%% If your work has an appendix, this is the place to put it.
\appendix
\end{document}
\endinput
%%
%% End of file `sample-acmsmall.tex'.
