%%
%% This is file `sample-acmsmall.tex',
%% generated with the docstrip utility.
%%
%% The original source files were:
%%
%% samples.dtx  (with options: `acmsmall')
%% 
%% IMPORTANT NOTICE:
%% 
%% For the copyright see the source file.
%% 
%% Any modified versions of this file must be renamed
%% with new filenames distinct from sample-acmsmall.tex.
%% 
%% For distribution of the original source see the terms
%% for copying and modification in the file samples.dtx.
%% 
%% This generated file may be distributed as long as the
%% original source files, as listed above, are part of the
%% same distribution. (The sources need not necessarily be
%% in the same archive or directory.)
%%
%%
%% Commands for TeXCount
%TC:macro \cite [option:text,text]
%TC:macro \citep [option:text,text]
%TC:macro \citet [option:text,text]
%TC:envir table 0 1
%TC:envir table* 0 1
%TC:envir tabular [ignore] word
%TC:envir displaymath 0 word
%TC:envir math 0 word
%TC:envir comment 0 0
%%
%%
%% The first command in your LaTeX source must be the \documentclass
%% command.
%%
%% For submission and review of your manuscript please change the
%% command to \documentclass[manuscript, screen, review]{acmart}.
%%
%% When submitting camera ready or to TAPS, please change the command
%% to \documentclass[sigconf]{acmart} or whichever template is required
%% for your publication.
%%
%%
\documentclass[acmsmall]{acmart}

%%
%% \BibTeX command to typeset BibTeX logo in the docs
\AtBeginDocument{%
	\providecommand\BibTeX{{%
			Bib\TeX}}}

%% Rights management information.  This information is sent to you
%% when you complete the rights form.  These commands have SAMPLE
%% values in them; it is your responsibility as an author to replace
%% the commands and values with those provided to you when you
%% complete the rights form.
\setcopyright{acmlicensed}
\copyrightyear{2018}
\acmYear{2018}
\acmDOI{XXXXXXX.XXXXXXX}


%%
%% These commands are for a JOURNAL article.
\acmJournal{JACM}
\acmVolume{37}
\acmNumber{4}
\acmArticle{111}
\acmMonth{8}

%%
%% Submission ID.
%% Use this when submitting an article to a sponsored event. You'll
%% receive a unique submission ID from the organizers
%% of the event, and this ID should be used as the parameter to this command.
%%\acmSubmissionID{123-A56-BU3}

%%
%% For managing citations, it is recommended to use bibliography
%% files in BibTeX format.
%%
%% You can then either use BibTeX with the ACM-Reference-Format style,
%% or BibLaTeX with the acmnumeric or acmauthoryear sytles, that include
%% support for advanced citation of software artefact from the
%% biblatex-software package, also separately available on CTAN.
%%
%% Look at the sample-*-biblatex.tex files for templates showcasing
%% the biblatex styles.
%%

%%
%% The majority of ACM publications use numbered citations and
%% references.  The command \citestyle{authoryear} switches to the
%% "author year" style.
%%
%% If you are preparing content for an event
%% sponsored by ACM SIGGRAPH, you must use the "author year" style of
%% citations and references.
%% Uncommenting
%% the next command will enable that style.
%%\citestyle{acmauthoryear}


%%
%% end of the preamble, start of the body of the document source.
\begin{document}
	
	%%
	%% The "title" command has an optional parameter,
	%% allowing the author to define a "short title" to be used in page headers.
	\title{Uncomputation}
	
	%%
	%% The "author" command and its associated commands are used to define
	%% the authors and their affiliations.
	%% Of note is the shared affiliation of the first two authors, and the
	%% "authornote" and "authornotemark" commands
	%% used to denote shared contribution to the research.
	\author{Maisa Kirisame}
	\email{marisa@cs.utah.edu}
	\orcid{1234-5678-9012}
	\authornotemark[1]
	\affiliation{%
		\institution{University of Utah}
		\streetaddress{P.O. Box 1212}
		\city{Dublin}
		\state{Ohio}
		\country{USA}
		\postcode{43017-6221}
	}
	
	%%
	%% By default, the full list of authors will be used in the page
	%% headers. Often, this list is too long, and will overlap
	%% other information printed in the page headers. This command allows
	%% the author to define a more concise list
	%% of authors' names for this purpose.
	\renewcommand{\shortauthors}{Trovato et al.}
	
	%%
	%% The abstract is a short summary of the work to be presented in the
	%% article.
	\begin{abstract}
		Program execution need memory. Program may run out of memory for multiple reasons: big dataset, exploding intermediate state, the machine have less ram then others. When this happens, the program either get killed, or the operating system swaps, significantly degrading the performance.
		We proposing a technique, Uncomputation, that allow the program to continue running even after breaching the memory limit, without a significant performance degrade like the one introduced by swapping.
		Uncomputation work by turning computed values back into thunk, and upon needing the thunk, computing and storing them back.
		A naive implementation of uncomputation will face multiple problems. Among them, the most crucial and the most challenging one is breadcrumb. After a value is uncomputed, it's memory can be released but extra memory, breadcrumb, is needed to recompute the value back when needed.
		The smaller a value is, the breadcrumb it leave will take up a larger portion of memory, until the size of a value is less then or equal to that of the size of breadcrumb, in which uncomputing give no, or even negative memory benefit at all.
		The more value uncomputed, the more breadcrumb they leave behind, until most of the memory is used to record breadcrumb. This prohibit asymptotic improvement commonly seen in memory-cosntrainted algorithms.
		We present a runtime system, zombie, implemented as a library, that is absolved of the above breadcrumb problem, seemingly storing recompute information in 0-bits and violating information theory.
	\end{abstract}
	
	%%
	%% The code below is generated by the tool at http://dl.acm.org/ccs.cfm.
	%% Please copy and paste the code instead of the example below.
	%%
		
	%%
	%% Keywords. The author(s) should pick words that accurately describe
	%% the work being presented. Separate the keywords with commas.
	\keywords{Do, Not, Us, This, Code, Put, the, Correct, Terms, for,
		Your, Paper}
	
	\received{20 February 2007}
	\received[revised]{12 March 2009}
	\received[accepted]{5 June 2009}
	
	%%
	%% This command processes the author and affiliation and title
	%% information and builds the first part of the formatted document.
	\maketitle
	
	\section{Introduction}
	Program use memory.
	
	Program may not have enough memory.
	
	Multiple reason:
	Huge input - opening a file of multiple gigabytes will hang editors
	
	Large intermediate state - model checker, chess bots, etc
	
	Small machine - poor people, embedded device
	
	Even if there is enough memory, still good to use less!
	Free up memory to run other process/increase cache size/give more room to garbage collectors.
	
	\section{Problems}
	How to measure size of a Value? (There is sharing)
	
	Partially evict a datastructure - how to reconnect
	
	What value to evict
	
	Breadcrumb
	
	\section{API}
	Zombie provide a monadic api.
	In particular, we exposed a type constructor, Zombie: * -> *,
	with the following 3 operations:
	
	return: X -> Zombie X

	bindN: Zombie X... -> (X... $\sim$> Zombie Y) -> Zombie Y

	get: Zombie X -> X
	
	the type Zombie externally speaking, behave as the Identity Monad - return and get is the identity, and bindN is the reversed apply. however, zombie will uncompute value behind the scene, recomputing them when needed, to save memory.
	
	two things worth noting:
	
	0 - we have a bindN, which take multiple Zombie, and a static function that does not capture any variable, from the inner type to another Zombie. This is different from the usual bindN accepting a single M, alongside a closured argument. This is because a closure will capture non-zombie values, so memory cannot be returned. Imagine the following function: M A -> M B -> M (A * B). An implementation via bind and return will capture A, causing said A to be non-evictable. note that bindN is of the same power as a closure allowing monadic bind.
	
	1 - the get function should only be used to obtain output, that will become a Zombie (or is used in a function that produce Zombie) again. This is because get strip out all metadata zombie maintain, so it is less efficient then obtaining the value via bindN.
	
	\section{Implementation}
	
	\section{Proof}
	
	\section{Eval}
%% If your work has an appendix, this is the place to put it.
\appendix
\end{document}
\endinput
%%
%% End of file `sample-acmsmall.tex'.
