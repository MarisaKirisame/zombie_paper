%%
%% This is file `sample-acmsmall.tex',
%% generated with the docstrip utility.
%%
%% The original source files were:
%%
%% samples.dtx  (with options: `acmsmall')
%% 
%% IMPORTANT NOTICE:
%% 
%% For the copyright see the source file.
%% 
%% Any modified versions of this file must be renamed
%% with new filenames distinct from sample-acmsmall.tex.
%% 
%% For distribution of the original source see the terms
%% for copying and modification in the file samples.dtx.
%% 
%% This generated file may be distributed as long as the
%% original source files, as listed above, are part of the
%% same distribution. (The sources need not necessarily be
%% in the same archive or directory.)
%%
%%
%% Commands for TeXCount
%TC:macro \cite [option:text,text]
%TC:macro \citep [option:text,text]
%TC:macro \citet [option:text,text]
%TC:envir table 0 1
%TC:envir table* 0 1
%TC:envir tabular [ignore] word
%TC:envir displaymath 0 word
%TC:envir math 0 word
%TC:envir comment 0 0
%%
%%
%% The first command in your LaTeX source must be the \documentclass
%% command.
%%
%% For submission and review of your manuscript please change the
%% command to \documentclass[manuscript, screen, review]{acmart}.
%%
%% When submitting camera ready or to TAPS, please change the command
%% to \documentclass[sigconf]{acmart} or whichever template is required
%% for your publication.
%%
%%
\documentclass[acmsmall]{acmart}
\usepackage{mathpartir}

%%
%% \BibTeX command to typeset BibTeX logo in the docs
\AtBeginDocument{%
	\providecommand\BibTeX{{%
			Bib\TeX}}}

%% Rights management information.  This information is sent to you
%% when you complete the rights form.  These commands have SAMPLE
%% values in them; it is your responsibility as an author to replace
%% the commands and values with those provided to you when you
%% complete the rights form.
\setcopyright{acmlicensed}
\copyrightyear{2018}
\acmYear{2018}
\acmDOI{XXXXXXX.XXXXXXX}


%%
%% These commands are for a JOURNAL article.
\acmJournal{JACM}
\acmVolume{37}
\acmNumber{4}
\acmArticle{111}
\acmMonth{8}

%%
%% Submission ID.
%% Use this when submitting an article to a sponsored event. You'll
%% receive a unique submission ID from the organizers
%% of the event, and this ID should be used as the parameter to this command.
%%\acmSubmissionID{123-A56-BU3}

%%
%% For managing citations, it is recommended to use bibliography
%% files in BibTeX format.
%%
%% You can then either use BibTeX with the ACM-Reference-Format style,
%% or BibLaTeX with the acmnumeric or acmauthoryear sytles, that include
%% support for advanced citation of software artefact from the
%% biblatex-software package, also separately available on CTAN.
%%
%% Look at the sample-*-biblatex.tex files for templates showcasing
%% the biblatex styles.
%%

%%
%% The majority of ACM publications use numbered citations and
%% references.  The command \citestyle{authoryear} switches to the
%% "author year" style.
%%
%% If you are preparing content for an event
%% sponsored by ACM SIGGRAPH, you must use the "author year" style of
%% citations and references.
%% Uncommenting
%% the next command will enable that style.
%%\citestyle{acmauthoryear}

\usepackage{xcolor}
\usepackage{mdframed}
\newcommand\todo[1]{\textcolor{red}{#1}}
\newcommand\pavel{\todo{pavel look here} }

%%
%% end of the preamble, start of the body of the document source.
\begin{document}
	%%
	%% The "title" command has an optional parameter,
	%% allowing the author to define a "short title" to be used in page headers.
	\title{Uncomputation}
	%%
	%% The "author" command and its associated commands are used to define
	%% the authors and their affiliations.
	%% Of note is the shared affiliation of the first two authors, and the
	%% "authornote" and "authornotemark" commands
	%% used to denote shared contribution to the research.
	\author{Maisa Kirisame}
	\email{marisa@cs.utah.edu}
	\orcid{1234-5678-9012}
	\authornotemark[1]
	\affiliation{%
		\institution{University of Utah}
		\streetaddress{P.O. Box 1212}
		\city{Salt Lake City}
		\state{Utah}
		\country{USA}
		\postcode{43017-6221}
	}

	\author{Pavel Panchekha}
	\email{}
	\orcid{1234-5678-9012}
	\authornotemark[1]
	\affiliation{%
		\institution{University of Utah}
		\streetaddress{P.O. Box 1212}
		\city{Salt Lake City}
		\state{Utah}
		\country{USA}
		\postcode{43017-6221}
	}
	
	%%
	%% By default, the full list of authors will be used in the page
	%% headers. Often, this list is too long, and will overlap
	%% other information printed in the page headers. This command allows
	%% the author to define a more concise list
	%% of authors' names for this purpose.
	\renewcommand{\shortauthors}{Kirisame et al.}
	%%
	%% The abstract is a short summary of the work to be presented in the
	%% article.
	\begin{abstract}
		Program execution need memory. Program may run out of memory for multiple reasons: big dataset, exploding intermediate state, the machine have less memory than others, etc. When this happens, the program either get killed, or the operating system swaps, significantly degrading the performance.
		We propose a technique, uncomputation, that allow the program to continue running gracefully even after breaching the memory limit, without significant performance degradation.
		Uncomputation work by turning computed values back into thunk, and upon re-requesting the thunk, computing and storing them back.
		A naive implementation of uncomputation will face multiple problems. Among them, the most crucial and the most challenging one is that of breadcrumb. After a value is uncomputed, it's memory can be released but some memory, breadcrumb, is needed, so we can recompute the value back.
		Ironically, in a applicative language, due to boxing all values are small. This mean uncomputation, implemented naively, will only consume more memory, defeating the purpose.
		We present a runtime system, implemented as a library, that is absolved of the above breadcrumb problem, seemingly storing recompute information in 0-bits and violating information theory.
	\end{abstract}
	
	%%
	%% The code below is generated by the tool at http://dl.acm.org/ccs.cfm.
	%% Please copy and paste the code instead of the example below.
	%%
		
	%%
	%% Keywords. The author(s) should pick words that accurately describe
	%% the work being presented. Separate the keywords with commas.
	\keywords{Do, Not, Us, This, Code, Put, the, Correct, Terms, for,
		Your, Paper}
	
	\received{20 February 2007}
	\received[revised]{12 March 2009}
	\received[accepted]{5 June 2009}
	
	%%
	%% This command processes the author and affiliation and title
	%% information and builds the first part of the formatted document.
	\maketitle
	bullet points:
	\begin{enumerate}
		% Intro (2pg)
		\item The importance of saving memory
		\item Pros of recomputation
		\item The perlis of breadcrumbs (no idea how to make this positive)
		\item Our approach leave no breadcrumb. formally state Guarantee
		% Overview (3pg)
		\item taba: explain
		\item evicting a context
		\item multiple eviction and no-breadcrumb
		\item picture of eviction
		\item replaying the context
		\item recursive replaying
		\item picture of replaying
		% Core Idea (3pg)
		\item Ordering all value reference (pointer) by logical time (tock)
		\item Replay objects (checkpoints): thunk, input, storage
		\item Approximate data structure: return largest key <= k
		\item (Recursive) Replaying
		% Language
		\item Base Language (purely functional lambda calculus)
		\item abstract machine
		% Implementation (?pg)
		\item The approximate data structure (splay list)
		\item Bit counting
		\item The abstract Machine
		% Formal proof
		\item Double O(1)
		\item Correctness property
		% Cache Policy(2pg)
		\item Time and Space measurement
		\item Union Find
		\item GDSF
		% Optimization
		\item Unrolling
		% Eval
		\item Benchmark explanation
		\item Setup
		\item Result table
		\item Explain Result
		% Related Work (2pg)
		\item Related Work(2pg)
		\item Conclusion
	\end{enumerate}
	
	adapton bullet points:
	% core contribution of paper: tying trace/patch/recomputation formally
	\begin{enumerate}
		% intro:
		\item abstract
		\item other work: IC(traces)
		\item cant do sharing/swapping/switching/interactive application
		\item key concept: demand computation graph
		\item formalization sneak peak
		\item some numbers
		\item spreadsheet example
		
		% overview: go backward!!!
		\item show ref/get/set/thunk/force and demand computation graph
		\item graph about sharing
		\item subconcept: dirtying the dcg
		\item why dirtying do swapping
		
		
		\item textual description of the calculus (CBPV)
		\item the changes to CBPV
		\item greeks definition
		\item describing trace
		\item greeks type
		\item greeks operational semantic
		\item text operational semantic
		\item text type
		
		\item patching a trace
		\item formal syntax of spreadsheet example
		\item API, and explaining memo
		\item the core change propagation algorithm
		\item explaining change propagation
		\item why change propagation better then other approach

		% eval
		\item explaining setup for lazy/swapping/switching/batch
		\item baseline and what is the machine setup
		\item table of result
		\item talk about some benchmark
		\item explaining result
		\item overhead
		\item spreadsheet example - explain what is a spreadsheet
		\item what it do to spreadsheet
		\item talk about number'
		
		% related work
		\item IC
		\item self adjusting computation
		\item FRP
		\item conclusion
	\end{enumerate}
	\section{Intro}
		\begin{table}
		\begin{center}
			\begin{tabular}{ |c|c| } 
				\hline
				Classical PL runtime & Zombie \\ 
				\hline\hline
				Address & Tock \\
				Value & Value \\
				Address Space & Tock Tree \\
				Constructing Values & Constructing Values and saving Thunks \\
				Reading from Address & Querying the Tock Tree \\
				Garbage Collection & Eviction \\
				Reading freed address: Impossible/Segfault & Recomputation \\
				\hline
			\end{tabular}
			\caption{Side by side comparison between classical PL runtime and Zombie}
		\end{center}
	\end{table}
	\section{Overview}	
	\section{Core Language}	
	Zombie works on a untyped, purely functional, call by value language. Program in the language then executed by the cek abstract machine.
	
	Below we sketch out the language and the cek machine. Note that this is the standard semantic - there is neither uncomputation nor replaying present. Uncomputation will be represented independently afterward.

	\begin{mdframed}
	\texttt{Name = N = A set of distinct names}
		
	\texttt{Expr = E := N | Left E | Right E | Case E N E N E | Prod E E | Zro E | Fst E}

	\texttt{| Lam N E | Let N E E | App E E}

	The source language
	\end{mdframed}
	
	\begin{mdframed}
		Continuation = K := Stop | KLeft K | KRight K | KCase Env N E N E K | KProd0 Env E K | KProd1 V K | KZro K | KFst K | KLet N Env E K | KApp0 Env E K | KApp1 V K
	
		\texttt{Environment = Env = (N, V)...}
		
		Value = V := VLeft V | VRight V | VProd V V | Clos Env N E
		
		State = Step Expr Env K | Apply K V

		step: (Expr, Env, K) -> (Expr, Env, K)

		step(N, Env, K) goto apply(K, Env(N))

		step(Left X, Env, K) goto step(X, Env, KLeft K)
		
		step(Right X, Env, K) goto step(X, Env, KRight K)
		
		step(Case X LN L RN R, Env, K) goto step(X, Env, KCase LN L RN R Env)
		
		step(Prod L R, Env, K) goto step(L, Env, KProd0 K R)
		
		step(Zro X, Env, K) goto step(X, Env, KZro K)
		
		step(Fst X, Env, K) goto step(X, Env, KFst K)
		
		step(Let A = B in C, Env, K) goto step(B, Env, KLet A K C Env)
		
		step(App f x, Env, K) goto step(f, KApp0 K x)
		
		step(Lam N E, Env, K) goto apply(K, Clos Env(fv)... N E)

		apply(KLeft K, V) goto apply(K, VLeft V)

		apply(KRight K, V) goto apply(K, VRight V)
		
		apply(KCase Env LN L RN R K, VLeft V) goto (L, Env(LN := V), K)

		apply(KCase Env LN L RN R K, VRight V) goto (R, Env(RN := V), K)
		
		apply(KProd0 Env R K, V) goto (R, Env, KProd1 V Env K)
		
		apply(KProd1 L K, V) goto apply(VProd L V, K)
		
		apply(KZro K, VProd X Y) goto apply(X, K)

		apply(KFst K, VProd X Y) goto apply(Y, K)
		
		apply(KLet A Env C K, B) goto (C, Env(A := B), K)
		
		apply(KApp0 Env X K, V) goto (X, Env, KApp1 V K)
		
		apply(KApp1 (Clos Env N E) K, V) goto (E, Env(N := V), K)

		Abstract Machine
	\end{mdframed}
	\section{Uncomputing and Recomputing}
	\subsection{Tock}
	The critical insight of zombie is that multiple abstract machine state compute the same value. To be more precise, if a machine state x step to a machine state y, x must have computed all value that y might compute, and possibly more. This indicate that we do not have to store all previous machine state - some might be dropped in favor of older ones.
	
	For this purpose, we introudced a global, logical time of 64 bit int, a tock. Tock start at 0, and increase by 1 on each transition(step/apply) in the abstract machine, and whenever a value is constructed. Conversely, all tock smaller then the current tock correspond to either a value, or a executed machine state, and vice versa.
	
	More importantly, given a value that correspond to tock X, any machine state with tock Y < X will recompute it. The largest Y under the constraint will do the least amount of transition computing said value.
	\subsection{Tock Tree}
	To pair a value with it's tock concretely(I mean in the runtime, the word look bad), and to allow a value to be recomputed, we abstract over the memory space(need better words), replacing pointers to value, to tocks instead. The actual values are stored on a global data structure, the tock tree. Reading from a pointer is replaced from querying the tock tree with the tock. The tock tree additionally store machine state as they are executed, so a value might be uncomputed and recomputed in the future with any earlier machine state.
	
	The tock tree is a binary search tree with the crucial property that lookup returns the largest node with key <= the input. This allow us to drop any node in the tock tree, with the exception of the leftmost node. Each node on the tock tree correspond to an execution of a transition, and contain:
	\begin{enumerate}
		\item The starting tock of the transition execution, t
		\item An array of cell(actual value), created during the transition, of corresponding tock t+1, t+2...
		\item The state it transit to.
	\end{enumerate} 
	Note that it store the transit-to state, but not the transit-from state, for that state is useless. Additionally, two tock is stored: the begin tock, t, and the end-tock, implicitly stored in the State. It is technically possible to compute the end-tock by adding t + 1 with length of the array.

	Uncomputing is then merely deleting a non-leftmost value from the tock tree.

	Formally speaking, 
	\begin{mathpar}
		before: Value = V := VProd V V | ...
		
		after: 
		Value = V = Tock

		Cell = Prod V V | ...
		
		before: 

		State = Step Expr Env K | Apply K V

		after:

		State = Step Expr Env K T | Apply K V T
		\end{mathpar}
	\subsection{Replay}
	During execution, the tock needed to be converted back to a Cell. 
	It proceed as follow:
	\begin{enumerate}
		\item to convert tock t to a Cell:
		\item query the tock tree on t to get a Node
		\item if the cell is in the array, return said Cell
		\item otherwise, issue a replay to t.
	\end{enumerate}
	A replay t suspend the current machine state, replacing it with the State in the Node returned from tock tree, and executing until the tock reach t. The old state is then resumed with the Cell at t.
	Replay form a stack: a replay might more replay.
	\begin{mathpar}
		Replay = R := NoReplay | Replaying T R
	\end{mathpar}
	\section{Implementation}
	\subsection{Tock Tree}
	To exploit the temporal/spatial locality, and the 20-80 law of data access (cite?), the tock tree is implemented as a slight modification of a splay tree.

	This design grant frequently-accessed data faster access time. Crucially, consecutive insertion take amortized constant time.
	
	The tock tree is then modified such that each node contain an additional parent and child pointer. The pointers form a list, which maintain an sorted representation of the tock tree. On a query, the tock tree do a binary search to find the innermost node, then follow the parent pointer if that node is greater then the key. This process is not recursive: the parent pointer is guaranteed to have a smaller node then the input key, as binary search will yield either the exact value, or the largest value less then the input, or the smallest value greater then the input.
	\subsection{Picking Uncomputation Candidate}
	Note that the guarantee we prove is independent of our policy that decide which value to uncompute (eviction policy).
	\subsubsection{Union Find}
	\subsubsection{The Policy}
	\subsubsection{GDSF}
	\subsection{Language Implementation}
	For implementation simplicity and interoperability with other programs, zombie is implemented as a C++ library, and the Cells are ref-counted. Our evaluation compiles the program from the applicative programming language formalized above(give name), to C++ code.
	\subsection{Optimization}
	\subsubsection{Fast access path}
	Querying the tock tree for every value is slow, as it requires multiple pointer traversal.
	To combat this, each Value is a Tock paired with a weak reference, serving as a cache, to the Cell. When reading the value, if the weak reference is ok, the value is return immediately. Otherwise the default path is executed, and the weak reference is updated to point to the new Result.
	\subsubsection{Loop Unrolling}
	To avoid frequent creation of node object, and their insertion to the tock tree, multiple state transition is packed into one.
	\subsection{Bit counting}
	\section{Formal Guarantee}
	\subsection{Safety}
	Evaluating under replay semantic give same result as under normal semantic
	\subsection{Liveness}
	Evaluating will eventually produce a value
	
	Decreasing on lexicalgraphic ordering on the replay stack do work
	\subsection{Performance}
	memory consumption is linear to amount of object with O(1) access cost
	\section{Evaluation}
	%% \subsection{Motivation}
% Program execution consume both space and time. While there are tons of research focused on reducing the time spent running a program, memory usage reduction had been consistently underappreciated.

% Yet, saving memory is still an important topic worth studying:
% \begin{itemize}
% 	\item Multi-tenancy. A end-user laptop execute multiple programs concurrently. As an example, a typical programmer might open an IDE to edit and compile code, Zoom for remote meeting, and additionally a browser with dozens of tabs open to lookup information on the internet. Among them, the worst is the web browser, as each tab is it's own separate process, with a renderer, a rule engine, along side a JavaScript runtime. All software above consume a decent amount of ram, and contend between themselves for memory.
% 	\item Huge input. While a typical image might have a resolution of around 1000 x 1000, large images might reach a size 100 to 1000 time larger than that. Images editors and viewers typically thrash or oom upon processing images of such size. Similarly, text editors typically process text of < 1MB, and fail on logs or other large text file reaching GBs. Likewise IDEs will parse and analyze the source code of a project, then stored the analyzed results for auto completion. Upon working on huge project such IDEs will crash.
% 	\item Intermediate state. A modern computer can read and write memory at a speed of GB per seconds. Without memory reclamation technique such as garbage collection, memory will run out at a matter of seconds. However, some applications is essentially out of reach from garbage collection, and must keep most if not all intermediate states around. These applications include time travelling debugger, jupyter notebook, reverse mode automatic differentiation for deep learning, incremental computation/hash consing, and algorithm that use search, such as model checker and chess bot.
% 	\item Low memory limit. Wasm have a memory limit of 2GB, and embedded device or poor people may have device with lower ram. GPU and other accelerator use their own separate memory and usually is of smaller capacity then main memory.
% \end{itemize}

% A typical and generic solution to reduce memory consumption is by uncomputation. Uncomputation trade space for time, by tagging values with the metadata that created it(thunk), and when memory is low, deallocate values which might be used later on, recomputing them back with the thunk when the value is needed again.

% Another generic solution is swapping, where upon low memory, pages are swapped to disk and swapped back when needed again. While swapping had been implemented for basically all operating systems, it is an especially bad solution for functional and object-oriented languages, as those languages allocate and deallocate lots of small object, yet swapping work on the granularity of pages, unable to separate out the hot objects, that had been recently accessed from the code objects, that had not been accessed.

% Typically, uncomputation is implemented in an ad-hoc, case by case basis. When a software consume more memory than desired, it's programmer can (1) look at the program, (2) decide which part is using excessive memory, then (3) add custom data type to record the thunk, alongside the code to replace it (4) implement a cache eviction policy to decide what data to evict(uncompute). Another solution is to analyze the algorithm carefully, replacing the algorithm with a specialized version with recomputation in mind, see gradient checkpointing/iterative deepinging depth first search/island algorithm.

% Needless to say, this process is extremely cumbersome, taking precious developers time away from more critical task such as optimizing cpu usage or implementing new features, so a generic, automatic approach is needed to reduce memory consumption.

% \subsection{Failed Attempt}
% On first glance, uncomputation is easy, as it's cousin, lazy evaluation, is heavily studied. In lazy computation, object might be represented as thunks, which, when needed, is then computed, and the thunk is replaced by the value. A solution to uncomputation is to modify thunk, so that, upon computing, the old function is still kept. This allow the thunk to uncompute as if it had never been recomputed.

% \begin{mathpar}
% 	$Lazy Evaluation: type 'a lazy = MkLazy of ('a, unit -> 'a) either ref \\
% 	Uncomputation: type 'a uncompute = MkUncompute of 'a option ref * (unit -> 'a)$
% \end{mathpar}

% Just like lazy evaluation, we can then uniformally lift all values on top of our modified value type, inserting code that force weak head normal form upon data access, and have some cache policy to decide what to uncompute, which merely rewrite the reference back into the empty optional value.

% However, there are multiple problems with this approach:
% \begin{enumerate}
% 	\item size measurement. We want to gather statistics to decide what values to evict, and one of the most important statistics is the memory a object consume. However, the heap form a complex object graph, and fast cardinality estimation is highly complex.
% 	\item dopple ganger. The translation will lift a list into nested uncompute. Now suppose variable X hold a list, while variable Y hold a sublist of X, sharing the same representation\todo{bad wording dont know how to fix}. When X is uncomputed and recomputed, the corresponding Y part will be duplicated, so there will be two representation of Y in memory. In the extreme case, such duplication can force the program to take exponentially more memory. 
% 	\item bread crumb. A thunk capture other value as free variable, which contain a thunk part, capturing more value. This recursive process capture all value that the current value transitively compute on. Since we still need memory to store the thunk itself, and since a thunk is typically larger then an object, anything we gained on uncomputation, will be offset by the metadata.
% \end{enumerate}

% While the above three problems seems difficult, we can observe they are all but mere manifestation of recursiveness. In particular, size measurement and dopple ganger deal with recursive objects, and bread crumb deal with recursive thunk. This indicate that a solution that handle recursiveness well naturally solve the three problems at once.

% \subsection{A recursive Solution}
% To combat recursiveness we take heavy inspiration from Abstracting Abstract Machine (AAM). AAM lift abstract interpretation onto programming languages with arbitrary feature. The critical problem there is likewise recursiveness, as a naive lifting might allow the lattice infinite merge, causing the analysis to non-terminate. AAM propose to use an abstract machine that abstract over pointers, allocations and lookups to manage recursiveness.

% Likewise, the three recursiveness problem listed above can be handled by a similar manner. We started from a purely functional language, specifying it's semantic with the CEK machine. We can then abstract over pointers/allocations/lookups similarly.

% Most importantly, our key insight is that by a careful handling of our core data structure, the tock tree, which will be explained later, we can remove a value, alongside the metadata(thunk) on that value, yet still recompute it later. This solve the breadcrumb problem which render the naive apporach unsalvageable.

% Alongside our solution is proof that it is both correct and efficient. Our correctness proof state that uncomputing will not effect the final result(preservation), and will not infinite loop(progress). What get computed by the raw, uncompute-free semantic, no matter what we evicted. Our efficiency proof state that if our memory consumption is O(n), where n is the amount of live objects: objects that take O(1) time to force into weak head normal form. In particular, this imply asymptotically we use no space to store any evicted object, yet we somehow can access the metadata that recompute them!

% We had implemented our solution, alongside a eviction policy that is both fast to compute, and make better decision over classical eviction policy like LRU, random, and GDSF.

\subsection{Motivation}
Program execution consumes both space and time. While there is tons of research focused on reducing the time spent running a program, memory usage reduction has been consistently underappreciated.

Yet, saving memory is still an important topic worth studying:
\begin{itemize}
    \item Multi-tenancy. An end-user laptop executes multiple programs concurrently. As an example, a typical programmer might open an IDE to edit and compile code, Zoom for remote meetings, and additionally a browser with dozens of tabs open to look up information on the internet. Among them, the worst is the web browser, as each tab is its own separate process, with a renderer, a rule engine, and a JavaScript runtime. All software above consume a decent amount of RAM and contend between themselves for memory.
    \item Huge input. While a typical image might have a resolution of around 1000 x 1000, large images might reach a size 100 to 1000 times larger than that. Images editors and viewers typically thrash or oom upon processing images of such size. Similarly, text editors typically process text of < 1MB, and fail on logs or other large text files reaching GBs. Likewise, IDEs will parse and analyze the source code of a project and then store the analyzed results for auto-completion. Upon working on a huge project such IDEs will crash.
    \item Intermediate state. A modern computer can read and write memory at a speed of GB per second. Without memory reclamation techniques such as garbage collection, memory will run out in a matter of seconds. However, some applications are essentially out of reach from garbage collection and must keep most if not all intermediate states around. These applications include time traveling debugger, Jupyter Notebook, reverse mode automatic differentiation for deep learning, incremental computation/hash consing, and algorithms that use search, such as model checker and chess bot.
    \item Low memory limit. Wasm has a memory limit of 2GB, and embedded devices or poor people may have devices with lower RAM. GPU and other accelerator use their own separate memory and usually are of smaller capacity than main memory.
\end{itemize}

At best, programmers today approach memory savings
  in an ad-hoc, case-by-case way:
  they examine their program
  and rewrite it to use less memory,
  possibly by using caching, lazy computation, or novel algorithms.
However, these case-by-case solutions
  are not always available,
  since they depend on the details of the computation.
Moreover, even when available, 
  such solutions can require substantial programmer effort,
  including rewriting the program data flow
  to enable new algorithms or architectures
  that reduce memory usage.

We instead seek a generic and automatic solution to the problem
  of reducing program memory consumption.
By generic, we mean a solution
  that is applicable to arbitrary programs
  in a given programming language.
By automatic, we mean a solution
  that requires no programmer annotations
  or other modifications to the program itself.
In other words, we seek a runtime system
  that would automatically reduce the memory consumption
  of a program.
Moreover, we would like to reduce memory consumption
  below the level of live heap memory
  (unlike garbage collection)
  that adapts to fine-grained program behavior
  (unlike swapping.)

\name is a generic and automatic runtime system
  that reduces program memory consumption
  by evicting values from memory
  and recomputing them when necessary.
That is, \name is a runtime
  for a simple purely-functional program language
  (which we call $\lambda_Z$)
  that pairs every value on the heap
  with the information necessary
  to recompute that value if it were evicted.
\name can then evict any heap value
  at any point in time,
  thereby reducing program memory consumption,
  working below the level of live heap memory
  and adapting to fine-grained program behavior.
In doing so,
  \name preserves the original program semantics,
  makes forward execution progress
  and is relatively efficient in its use of memory.
Of course, the cost of replay
  is longer program runtime
  (since individual program steps may be replayed many times);
  in this way, \name introduces a space-time tradeoff,
  where programs run faster when more memory is available.
That said, various optimizations,
  such as batching, a fast path for hot objects,
  and union find for fast eviction,
  allows \name to run at moderate overhead
  compared to a more traditional runtime.

Moreover, \name achieves its memory reduction
  without adding a commensurate memory overhead.
That is, the amount of metadata stored by \name
  to enable recomputation
  is proportional to the total memory,
  and independent of how long the program has been running.
To achieve this,
  \name stores replay points sparsely,
  so that replaying a given computation
  might involve replaying earlier computations as well,
  starting from whatever earlier replay point is available.
The sparse replay points
  allow \name to keep the total memory dedicated to replay limited
  while still enabling it to recompute any arbitrary value.

To evaluate \name,
  we test 10 $\lambda_Z$ programs
  that implement common algorithmic tasks
  such as balanced binary trees, list and array manipulation,
  and recursive functions.
Each program is run with different memory limits,
  ranging as low as \todo{one thousandth}
  of a traditional run-time's memory consumption.
\name allows the programs to run
  despite severe memory restrictions,
  with a $10\times$ reduction in memory
  typically resulting in a $5\times$ increase in running time.

In short, this paper contributions:
\begin{itemize}
\item Zombie, a runtime that saves memory
  for arbitrary programs without programmer effort;
\item Proofs that Zombie
  preserves program semantics, makes progress, and is efficient;
\item A collection of optimizations
  that enable Zombie to run with moderate runtime overhead.
\end{itemize}

	%\section{Uncomputation(5pg)}
\subsection{Motivation}
\subsection{Strawman}
\subsubsection{API}
\todo{unable to use the monadic api.}
\subsubsection{Semantic}
\subsection{Guarantee}
\subsection{Problems}
While the high-level idea seems straightforward, there are multiple subtle questions: 
\begin{itemize}
	\item Recursiveness. A value, for example, a list, might be recursive. Furthermore, there might be sharing inside such datastructure. In such a case, how do we measure the size of a value, which is a valuable statistics guiding what to uncompute?
	\item Partialness. Suppose a list is generated via anamorphism. We might want to uncompute the head of the list, but keeping some intermediate node inside the list.
	When recomputing said list, we need to be able to retrieve such intermediate node, to
	avoid spending extra time to recompute them, and extra memory to store them twice.
	\item Uncomputation candidates. Which value should we pick to uncompute?
	\item Breadcrumb. After a value is uncomputed, we need to store information needed to recompute said value. This become a bottleneck when most value is uncomputed, or when each value is small.
\end{itemize}

We present Zombie, a library for uncomptuation, alongside solutions to the questions above.
\subsubsection{recursiveness}
\subsubsection{doppelganger}
\subsubsection{breadcrumb}

	%\section{Plant vs Zombie(8pg)}
Combining the 3 insights above we now sketch an implementation of Zombie.
There is a global tock, which value begin from 0, increasing on every invocation of bind/return.
A Zombie<X> hold an Integer, tock, instead of value of type X.
The value is stored in the tock tree.
Whenever a call to return or bind is finished, we put a Node onto the tock tree.
The range of said node is the tock value at the begin and end of the invocation.

For return, the node contain the value.

For bind, the node contain the list of input Zombie, the output Zombie, and the function pointer.

In bind, we have to force values from Zombie X to X. This is done via looking it up from the tock tree. If the look up node is not a Value, but the Thunk, we need to rewind the tock to the beginning of said thunk, replay the value, get the value from the tock tree (todo: avoid infinite loop here), and restore the tock.

\section{Tock Tree API}
we need a data structure, such that when after removing a node, looking said node up does not return Nothing, but rather return a less precise answer.

Below is an API summing up what we need.

Tock = u64

Range = Tock * Tock

Make: TockTree X

Insert: TockTree X -> Range * X -> ()

Per two call to Insert, their range must either be nested or non-overlapping.

Lookup: TockTree X -> Tock -> Option (Range * X)

Remove: TockTree X -> Range -> ()

geometric series
\subsection{OLD}
All Zombies are threaded through a logical clock, a tock, stored as a u64.
During execution, the clock increases whenever makeZombie and bindZombie are called.

A Zombie contains, theoretically speaking, contains only the tock that the Zombie is created at.
To access the actual value of the Zombie, a ZombieNode, we rely on a global data structure.
It will be explained later.
(We also store a weakpointer to ZombieNode as a cache, to quicken access. This is just a cache, and is irrelevant to the rest of Zombie).
A ZombieNode holds X but is inherited from the Object class, which contain a virtual destructor and nothing else. This essentially allow one to type erase ZombieNode of different type, to put them into the same data structure.

We implement Zombie via two global data structures, a log and a pool.
The pool track all currently in-memory Zombie representation - it manage space
The log record actions taken used for recomputation - it manage time

The pool holds a vector of uniqueptr<Phantom>, the class that manage eviction.
It has api to evict objects, and api to score the profitability of eviction.
When we are low on memory, we can find a value in this vector, call evict(), and remove it.
This will free up memory in the log.

For every call to makeZombie and bindZombie, we keep track of when the function is called vs when the function exited. This establishes a tock range. 
For makeZombie, since it does not call any more functions, its interval length is 1, containing exactly the tock of the zombie.
Tock range overlap iff one contain another, iff one function calls another function. We then store tock range, paired with information on said function. 
For makeZombie we store a shared pointer to the ZombieNode. 
For bindZombie, we record the function used to compute it, as std::function<void(const std::vector<const void*>)>, alongside the tocks of all input Zombies, and the tock of output Zombie. We call it a Rematerializer. The input argument is type erased to void*, as Zombie is type polymorphic, but we need a type uniform interface here.

The log is some kind of interval tree. Each node in the tree stores its begin and end interval, alongside a value, and a BST from tock to sub nodes. The key of BST is the start of the sub node’s interval. When we query the log via a tock, we will find a value with the closest interval containing said tock. This mean, we might query for a Zombie, but instead of getting a ZombieNode, we get a Rematerializer.
In such a case, we replay the Rematerializer by fetching the ZombieNode from the tock (replaying recursively if necessary), setting the clock to begin of the function temporarily, and entering said function. Afterward we set the clock back, and fetch the value from a tree.
When we set the clock via rematerialization, we always set it to a smaller value, thus guaranteeing termination.

The log itself, when replaying, also serve as a memo table, indexed by tock. When we want to create a Zombie, we will skip when such Zombie already exist in the log. When we want to run BindZombie, we return the result tock when the precise Rematerializer exist in the log. This allow us to not compute the same function multiple time, if we know its result, and allow us to not store the same Zombie multiple time.

Note that all non-top-levels node in the tree can be removed. When we do so, we will free up memory, and when we need the value again, we can always replay a node higher up.

This design allow us to store n log n amount of metadata per n alive(nonevicted) objects.

Thus, when a Rematerializer dominate no value, and another value dominate Rematerializer, we remove it. (todo: implement). Idea: manage this using pool.

There is a slight error with the above design: maybe the value is recomputed and put on the tree, but then evicted, and fetch will not return a ZombieNode but a Rematerializer. To fix this we introduce a global data structure called Tardis, consist of a tock and a shared ptr<ZombieNode>*. When we create Zombie, when the tock match that of tardis, it will write to the shared ptr, holding the value longer.

\subsection{replaying semantic}
Program evaluate differently in replaying mode then in non-replay mode, as relevant Node might already be on the tock tree. When 

Every entering tock (lhs of a range) uniquely identify a return/bind.

% see https://www.khoury.northeastern.edu/home/wand/csg264/latex/mathpartir/mathpartir.pdf
\begin{mathpar}
	A-Formula \and
	Longer-Formula \and
	And \and The-Last-One
	
	\inferrule
	{aa \\\\ bb}
	{dd \\ ee \\ ff}
	
	\inferrule*
	{\inferrule* {aa \\ bb}{cc}
		\\ dd}
	{ee}
\end{mathpar}


\subsection{Tock Tree}
\subsection{Recomputation}
\subsection{Correctness Gurantee}
\subsubsection{Progress}
\subsubsection{Preservation}
\subsection{Asymptotic}

	%\section{Call Stack}
While the above construction can 
\subsection{Tail Call}
Call stack - get a node with inf right value onto tock tree, fix when 'actual' return

Clock stack - put the value on the node

C++ stack - trampoline

Replay stack - bounded by the tock tree size * intermediate node count, so it is linear w.r.t. tock tree.
\subsection{Non Tail Call}
By doing a continuation passing style transform all call become tail call.
The continuation might grow in size but we can zombified it.
	%\section{Cache Policy(2pg)}
note that the cache policy does not effect the correctness of the implementation, only the performance.

GDSF
\subsection{Measuring compute time}
measuring time is more complex then it seems.

need to be recursive and uncount recursive time.
\subsection{recursive cache policy}
\subsection{union find}
	%\section{Eval(3pg)}
\subsection{Small Program}
List Program

Pascal Triangle(2d, 3d)

Interpreter for LC

PE for LC

Interpreter for IMP

Compiler from IMP to LC

String Serialization and Deserialization

String Compression and Uncompression

Parsing and Unparsing

String = radix-tree/flat-array/linked-list

KD-tree, on shortest distance and on ray tracing

statistic \todo{ask}

Graph - DFS BFS MCTS

\todo{where does undo redo fit in?}
\subsection{Composite Program}
compose the programs to form larger program
	%\section{Case Study(2pg)}
	%\section{Related Work(2pg)}
\subsection{Memory-Constrained Algorithm}
Memory Limit is a common problem. Lots of different subfields of CS had experienced such problem, and developed specialized algorithm that trade space for time, by recomputing, for their own need. \todo{cite}. In particular, a famous algorithm, TREEVERSE, had been re-discovered independently multiple times in unrelated subfields.

Keeping uncomputation at each subfield not only cause multiple re-discovery and re-implementation, wasting valuable researcher/programmer time, but also come short in integration. Suppose a complex, memory-hungry software have 2 sub-parts, both of which are memory hungry. How much uncompute responsibility should each part take? What happens if two part depend on each other?
\subsection{Pebbling}
Space time tradeoff is a known-topic in Theoretical Computer Science. It is mostly modeled as a "pebble game" \todo{quickly explain}. It had been known that this problem is NP-complete and inapproximable. There are also generic result in this space. Hopcroft prove that for any program that take N time, it could be modified to take O(N / log(N)) space. However, the proof is non-constructive, and thus does not help building an automatic runtime like ours.

Our work mostly focus on the overhead problem - how to minimize the overhead of recording metadata needed for recomputation. How to make an API that is generic yet embeddable. In another words - our work is orthgonal to advances in pebble game. In fact, one could imagine replacing our cache policy with some of those algorithms.

We also provide a greedy cache policy that is both fast and make reasonably good decision on the list of benchmark, both real and synthetic, provided below, since the theoretical result is weak and non-constructive. There are some specialized result of pebble game on planar graph and on tree. One could imagine upon detecting planar graph/tree, switching from our cache policy to the theoretically optimal ones.
\subsection{Memoization}
Uncomputation is the dual to Memoization.
\subsection{Garbage Collection}
GC collect provably dead value, but we can also collect not-provably dead value, and even alive value.

It is like optimistic lock vs pessmistic lock.
\subsection{Compression}
Uncomputation is, fundamentally speaking, a form of compression.

Pigeon hole principle mean it is fundamentally impossible to have a generic compression algorithm. Uncomputation exploit the inductive bias that a program pointer is smaller then its memory consumption is larger.

	%\section{Constant Optimization}
%% If your work has an appendix, this is the place to put it.
\appendix
\end{document}
\endinput
%%
%% End of file `sample-acmsmall.tex'.
